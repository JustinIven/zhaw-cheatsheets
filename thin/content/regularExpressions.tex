\section{Reguläre Ausdrücke}
Reguläre Ausdrücke sind ein formale Sprache zur endlichen Beschreibung
(möglicherweise unendlicher) Sprachen über einem Alphabet $\Sigma$.

\subsection{Syntax}
Die Menge $\text{RA}_\Sigma$ der regulären Ausdrücke über $\Sigma$ ist
induktiv definiert durch:
\begin{itemize}
    \item $\varnothing, \varepsilon \in \text{RA}_\Sigma$
    \item \(\Sigma \subset \text{RA}_\Sigma\)
    \item \(R \in \text{RA}_\Sigma \Rightarrow (R^*) \in \text{RA}_\Sigma\)
    \item \(R,S \in \text{RA}_\Sigma \Rightarrow (RS) \in \text{RA}_\Sigma\)
    \item \(R,S \in \text{RA}_\Sigma \Rightarrow (R \mid S) \in \text{RA}_\Sigma\)
\end{itemize}

Die Menge \(\text{RA}_\Sigma\) der regulären Ausdrücke über dem Alphabet \(\Sigma\) ist eine Sprache über dem Alphabet \(\Sigma \cup \{\varepsilon, \varnothing, (, ), ^*, \mid\}\).

\subsubsection{Abkürzungen}
\begin{align*}
    R^+             & := RR^*                    \\
    R?              & := (R \mid \varepsilon)    \\
    [R_1,\dots,R_k] & := R_1 \mid \dots \mid R_k
\end{align*}


\subsection{Semantik}
Jedem regulären Ausdruck $R$ wird eine Sprache $L(R)$ zugeordnet:

\begin{align*}
    L(\varnothing) & = \emptyset                            \\
    L(\epsilon)    & = \{\varepsilon\}                      \\
    L(a)           & = \{a\} \quad \text{für } a \in \Sigma \\
    L(R^*)         & = L(R)^*                               \\
    L(RS)          & = L(R)L(S)                             \\
    L(R \mid S)    & = L(R) \cup L(S)
\end{align*}
Eine Sprache $A \subseteq \Sigma^*$ heisst \emph{regulär}, falls es einen
regulären Ausdruck $R$ mit $A = L(R)$ gibt.

\subsection{Rechenregeln}

Fürjedes Alphabet \(\Sigma\) und alle regulären Ausdrücke \(R,S,T \in \text{RA}_\Sigma\) gelten die folgenden Identitäten:

\begin{align*}
    L(R \mid S)          & = L(S \mid R)          \\
    L(R \mid (S \mid T)) & = L((R \mid S) \mid T) \\
    L(R(S \mid T))       & = L(RS \mid RT)        \\
    L(R \mid R)          & = L(R)                 \\
    L(R(ST))             & = L((RS)T)             \\
    L((R^*)^*)           & = L(R^*)
\end{align*}
