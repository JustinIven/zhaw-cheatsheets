\section*{Alphabete, Wörter und Sprachen}

\subsection{Alphabet}
Ein Alphabet ist eine endliche, nichtleere Menge von Symbolen. Bsp.:
\[
    \Sigma = \{a,b,c\}, \quad
    \Sigma_\text{bool} = \{0,1\}, \quad
    \Sigma_\text{lat} = \{a,b,c \cdots ,z\}
\]

\subsection{Wörter}
Ein Wort über einem Alphabet ist eine endliche Folge von Symbolen aus diesem Alphabet. Das leere Wort wird mit $\varepsilon$ bezeichnet.
\[
    w = (x_1, x_2, \dots, x_n), \quad x_i \in \Sigma, \quad n \in \mathbb{N}
\]

Die Länge eines Wortes $w$ ist die Anzahl der Symbole, also $|w| = n$. Die absolute Häufigkeit eines Symbols $a$ in einem Wort $w$ wird mit $|w|_a$ bezeichnet. Zusätzlich gilt:
\[
    |\varepsilon| = |\varepsilon|_a = 0, \quad a \in \Sigma
\]

\subsubsection{Spiegelung und Palindrome}
Mit \(w^R\) wird das Spiegelwort von \(w\) bezeichnet, also die Umkehrung der Symbolfolge.
\[
    w^R = (x_1, x_2, \dots, x_n)^R = (x_n, x_{n-1}, \dots, x_1)
\]
Wenn \(w=w^R\) gilt, dann ist \(w\) ein Palindrom.

\subsubsection{Wörter der Länge \(k\)}
Die Menge aller Wörter der Länge \(k\) über einem Alphabet \(\Sigma\) wird mit \(\Sigma^k\) bezeichnet:
\[
    \Sigma^k = \{ w \in \Sigma^* \mid |w| = k \}
\]
Unabhängig von \(\Sigma\) gilt stets \(\Sigma^0 = \{\varepsilon\}\).

\subsubsection{Kleenesche Hülle}
Die Menge aller Wörter über einem Alphabet \(\Sigma\) wird mit \(\Sigma^*\) bezeichnet:
\[
    \Sigma^* = \bigcup_{k \ge 0} \Sigma^k
\]
Die Menge aller nichtleeren Wörter über \(\Sigma\) wird mit \(\Sigma^+\) bezeichnet:
\[
    \Sigma^+ = \Sigma^* \setminus \{\varepsilon\}
\]

\subsubsection{Teilwortrelationen}
\begin{description}
    \item[Infix] \(v\) ist Infix von \(w\) \(\iff\) Es existieren \(x,y \in \Sigma^*\) mit \(w = xvy\).
    \item[Präfix] \(v\) ist Präfix von \(w\) \(\iff\) Es existiert \(y \in \Sigma^*\) mit \(w = vy\).
    \item[Suffix] \(v\) ist Suffix von \(w\) \(\iff\) Es existiert \(x \in \Sigma^*\) mit \(w = xv\).
\end{description}
Man spricht von einem \emph{echten} Infix/Präfix/Suffix, wenn \(v \neq w\) gilt.


\subsubsection{Konkatination}
Die Konkatenation von zwei Wörtern \(x\) und \(y\) ist die Aneinanderreihung der Symbole von \(x\) gefolgt von den Symbolen von \(y\):
\[
    x \circ y = xy = (x_1, x_2, \dots, x_n, y_1, y_2, \dots, y_m)
\]
Die Länge der Konkatenation ist die Summe der Längen der beiden Wörter:
\[
    |xy| = |x| + |y|
\]
Die Konkatenation mit dem leeren Wort \(\varepsilon\) hat keine Auswirkung auf das Wort:
\[
    w\varepsilon = \varepsilon w = w
\]

\subsubsection{Wortpotenzen}
Die \(n\)-te Potenz eines Wortes \(x\) wird für alle \(n \in \mathbb{N}\) definiert als
\begin{align*}
    x^0 &:= \varepsilon \\
    x^{n+1} &:= x^n x
\end{align*}
Die Potenzierung mit der Kleeneschen Hülle ergibt die gleiche Menge:
\[
    (A^*)^* = A^*
\]


\subsection{Sprache}
Eine Teilmenge \(L \subseteq \Sigma^*\) von Wörtern heisst Sprache über dem Alphabet \(\Sigma\).
Es gelten die folgenden Eigenschaften:
\begin{itemize}
    \item \(\{\} = \emptyset\) ist die leere Sprache für jedes Alphabet \(\Sigma\).
    \item \(\{\varepsilon\}\) ist die Sprache, die nur das leere Wort enthält, für jedes Alphabet \(\Sigma\).
    \item \(\Sigma^*\) ist die Sprache aller Wörter über \(\Sigma\).
\end{itemize}

\subsubsection{Konkatenation}
Die Konkatenation von Sprachen \(A\) und \(B\) ist definiert als die Menge aller Konkatenationen von Wörtern aus \(A\) mit Wörtern aus \(B\):
\[
    AB = \{ uv \mid u \in A, v \in B \}
\]
Die Konkatenation von Sprachen ist assoziativ, aber im Allgemeinen nicht kommutativ:
\[
    (AB)C = A(BC), \quad AB \neq BA
\]
Ist \(A\) eine Sprache über \(\Sigma\) und \(B\) eine Sprache über \(\Gamma\), so ist \(AB\) eine Sprache über \(\Sigma \cup \Gamma\).

\subsubsection{Kleenesche Hülle}
Die Kleenesche Hülle \(A^*\) einer Sprache \(A\) ist definiert als die Vereinigung aller Potenzen von \(A\):
\[
    A^* = \bigcup_{k \ge 0} A^k
\]
Dabei ist \(A^0 = \{\varepsilon\}\) und \(A^{k+1} = A^k A\).
Die Kleenesche Hülle erfüllt die Eigenschaft:
\[
    (A^*)^* = A^*
\]


\subsection{Entscheidungsproblem}
Sei eine Sprache \(L \subseteq \Sigma^*\) gegeben. Das Entscheidungsproblem für \(L\) besteht darin, für ein beliebiges Wort \(x \in \Sigma^*\) zu entscheiden, ob \(x\) in \(L\) enthalten ist oder nicht:
\[
    x \in \Sigma^* \mapsto
    \begin{cases}
        \text{JA} & \text{wenn } x \in L, \\
        \text{NEIN} & \text{wenn } x \notin L.
    \end{cases}
\]
