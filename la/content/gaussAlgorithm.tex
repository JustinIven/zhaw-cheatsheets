\section{Gauss-(Jordan-)Algorithmus}

\subsection{Erweiterte Koeffizientenmatrix}
LGS $A\vec{x} = \vec{b}$ mit $m$ Gleichungen, $n$ Unbekannten als erweiterte Matrix:
\begin{equation*}
    (A \mid \vec{b} \ ) = \left[
        \begin{array}{cccc|c}
            a_{11} & a_{12} & \dots  & a_{1n} & b_1    \\
            a_{21} & a_{22} & \dots  & a_{2n} & b_2    \\
            \vdots & \vdots & \ddots & \vdots & \vdots \\
            a_{m1} & a_{m2} & \dots  & a_{mn} & b_m
        \end{array}
        \right]
\end{equation*}

\subsection{Elementare Zeilenoperationen}
\begin{enumerate}
    \item $Z_i \leftrightarrow Z_j$ \quad (Vertauschen)
    \item $Z_i \leftarrow \lambda \cdot Z_i$, $\lambda \neq 0$ \quad (Skalieren)
    \item $Z_i \leftarrow Z_i + \lambda \cdot Z_j$ \quad (Addieren)
\end{enumerate}

\subsection{Zeilenstufenform (ZSF)}
Bedingungen für ZSF:
\begin{enumerate}
    \item Nullzeilen stehen unten
    \item Erste Nichtnulleintrag (Pivot) jeder Zeile ist 1
    \item Pivots liegen strikt rechts vom Pivot der Zeile darüber
\end{enumerate}

\textbf{Reduzierte ZSF (RZSF):} Zusätzlich: Über jedem Pivot nur Nullen.

\begin{equation*}
    \underbrace{\left[
            \begin{array}{cccc|c}
                1 & * & * & * & * \\
                0 & 1 & * & * & * \\
                0 & 0 & 1 & * & * \\
                0 & 0 & 0 & 0 & 0
            \end{array}
            \right]}_{\text{ZSF}}
    \underbrace{\left[
            \begin{array}{cccc|c}
                1 & 0 & 0 & * & * \\
                0 & 1 & 0 & * & * \\
                0 & 0 & 1 & * & * \\
                0 & 0 & 0 & 0 & 0
            \end{array}
            \right]}_{\text{RZSF}}
\end{equation*}

\subsection{Algorithmen}
\textbf{Gauss-Algorithmus:} Überführung in ZSF durch Vorwärtselimination.

\textbf{Gauss-Jordan-Algorithmus:} Überführung in RZSF durch Vorwärts- und Rückwärtselimination.

\subsection{Lösbarkeit}
Rang: $\text{rg}(A) = $ Anzahl Nicht-Nullzeilen in ZSF von $A$

\begin{itemize}
    \item $\text{rg}(A) < \text{rg}(A \mid \vec{b})$: Keine Lösung (inkonsistent)
    \item $\text{rg}(A) = \text{rg}(A \mid \vec{b}) = n$: Eindeutige Lösung
    \item $\text{rg}(A) = \text{rg}(A \mid \vec{b}) < n$: Unendlich viele Lösungen, Parameter: $n - \text{rg}(A)$
\end{itemize}

\subsection{Parameterdarstellung der Lösung}
Bei unendlich vielen Lösungen ($\text{rg}(A) < n$):

%TODO: Beispiel einfügen
\textbf{Vorgehen:}
\begin{enumerate}
    \item RZSF bestimmen
    \item Freie Variablen (ohne Pivot) als Parameter wählen: $t_1, t_2, \dots$
    \item Gebundene Variablen (mit Pivot) durch freie Variablen ausdrücken
    \item Lösung: $\vec{x} = \vec{x}_0 + t_1 \vec{v}_1 + t_2 \vec{v}_2 + \dots$, mit $t_i \in \mathbb{R}$
\end{enumerate}
