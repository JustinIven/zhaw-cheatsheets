\section{Matrix Basics}
Ein Matrix (\(n \times m\)) ist eine rechteckige Anordnung von Elementen, die in \(n\) Zeilen und \(m\) Spalten organisiert ist.
\begin{equation*}
A = \begin{bmatrix}a_{11} & a_{12} & \cdots & a_{1m} \\
a_{21} & a_{22} & \cdots & a_{2m} \\
\vdots & \vdots & \ddots & \vdots \\
a_{n1} & a_{n2} & \cdots & a_{nm} \end{bmatrix}
\end{equation*}

Die Matrix (\(n \times 1\)) wird als Spaltenvektor bezeichnet, während die Matrix (\(1 \times m\)) als Zeilenvektor bezeichnet wird.

\subsection{Matrixoperationen}
\subsubsection{Addition und Subtraktion}
Zwei Matrizen \(A\) und \(B\) der gleichen Dimension (\(n \times m\)) können addiert werden:
\begin{equation*}
C = A + B \quad \text{mit} \quad c_{ij} = a_{ij} + b_{ij}
\end{equation*}
Subtraktion erfolgt analog:
\begin{equation*}
C = A - B \quad \text{mit} \quad c_{ij} = a_{ij} - b_{ij}
\end{equation*}

\subsubsection{Skalare Multiplikation}
Ein Matrix \(A\) kann mit einem Skalar \(k\) multipliziert werden:
\begin{equation*}
B = kA \quad \text{mit} \quad b_{ij} = k \cdot a_{ij}
\end{equation*}

\subsubsection{Matrixmultiplikation}
Zwei Matrizen \(A\) (\(n \times m\)) und \(B\) (\(m \times p\)) können multipliziert werden:
\begin{equation*}
C = AB \quad \text{mit} \quad c_{ij} = \sum_{k=1}^{m} a_{ik} b_{kj}
\end{equation*}
Die Matrixmultiplikation ist nicht kommutativ, d.h. \(AB \neq BA\) im Allgemeinen.

\subsubsection{Transponieren}
Die Transponierte einer Matrix \(A\) (\(n \times m\)) ist eine neue Matrix \(A^T\) (\(m \times n\)), bei der die Zeilen und Spalten vertauscht werden:
\begin{equation*}
A^T = \begin{bmatrix}a_{11} & a_{21} & \cdots & a_{n1} \\
a_{12} & a_{22} & \cdots & a_{n2} \\
\vdots & \vdots & \ddots & \vdots \\
a_{1m} & a_{2m} & \cdots & a_{nm} \end{bmatrix}
\end{equation*}
Die Transponierte hat folgende Eigenschaften:
\begin{itemize}
    \item \((A^T)^T = A\)
    \item \((A + B)^T = A^T + B^T\)
    \item \((kA)^T = kA^T\)
    \item \((AB)^T = B^T A^T\)
\end{itemize}