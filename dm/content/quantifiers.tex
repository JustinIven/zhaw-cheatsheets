\section{Quantoren}
Quantoren dienen zur Formalisierung von Aussagen wie:
\begin{itemize}
  \item $\forall x\,A(x)$: \emph{Für alle $x$ gilt $A(x)$}
  \item $\exists x\,A(x)$: \emph{Es existiert ein $x$ mit $A(x)$}
\end{itemize}

Mehrere gleichartige Quantoren:
$$\forall x,y\;A(x,y) \quad\text{statt}\quad \forall x\,\forall y\;A(x,y)$$


\subsection{Eingeschränkte Quantoren}
$$\forall x \in M\,A(x): \text{Für alle }x\in M \text{ gilt }A(x)$$
$$\exists x \in M\,A(x): \text{Es gibt }x\in M \text{ mit }A(x)$$

Auch möglich mit Relationen:
$$\forall x < y\,A(x) \quad \text{oder} \quad \exists x \le y\,A(x)$$

\subsection{Als Junktoren}
Für endliche Mengen $M = \{x_1, \dots, x_n\}$ gilt:
$$\forall x \in M\,A(x) \Leftrightarrow A(x_1)\land \dots \land A(x_n)$$
$$\exists x \in M\,A(x) \Leftrightarrow A(x_1)\lor \dots \lor A(x_n)$$

\subsection{Als Makros}
$$\exists x \in M\,A(x) \Leftrightarrow \exists x\,(x \in M \land A(x))$$
$$\forall x \in M\,A(x) \Leftrightarrow \forall x\,(x \in M \Rightarrow A(x))$$

\subsection{Zusammenhang mit Junktoren}
$$\neg \forall x\,A(x) \Leftrightarrow \exists x\,\neg A(x)
\quad\text{und}\quad
\neg \exists x\,A(x) \Leftrightarrow \forall x\,\neg A(x)$$
$$\forall x\,(A(x)\land B(x)) \Leftrightarrow (\forall x\,A(x)) \land (\forall x\,B(x))$$
$$\exists x\,(A(x)\lor B(x)) \Leftrightarrow (\exists x\,A(x)) \lor (\exists x\,B(x))$$

\subsection{Leere Quantoren}
Wenn $x$ in $B$ nicht vorkommt:
$$\forall x\,B \Leftrightarrow B, \quad \exists x\,B \Leftrightarrow B$$
