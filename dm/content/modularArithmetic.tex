\section{Modulare Arithmetik}

\subsection{Kongruenzrelation und Restklassen}
Für $n \in \mathbb{N}$ definiert man auf $\mathbb{Z}$ die Kongruenzrelation
\[
r \equiv_n s \;\Longleftrightarrow\; n \mid (r-s).
\]
Die Äquivalenzklasse von $z \in \mathbb{Z}$ heisst \emph{Restklasse} und wird mit
\[
[z]_n = \{z + kn \mid k \in \mathbb{Z}\}
\]
bezeichnet.
\begin{itemize}
    \item Abkürzend bezeichnet man \([z]_n\) mit \([z]\) oder \(\bar{z}\), wenn \(n\) aus dem Kontext klar ist.
    \item Jede ganze Zahl ist modulo $n$ eindeutig zu einer Zahl aus $\{0,\dots,n-1\}$ kongruent.
    \item Der Kleinste nicht negative Vertreter einer Restklasse $[z]_n$ wird als \emph{Kanonischer Vertreter} bezeichnet und ist gegeben durch $\operatorname{mod}(z,n)$.
\end{itemize}

\subsection{Rechnen mit Restklassen}
Die Menge der Restklassen modulo $n$ ist
\[
\mathbb{Z}/n = \{[0]_n,[1]_n,\dots,[n-1]_n\}.
\]
Addition und Multiplikation sind wohldefiniert durch
\[
[x]_n + [y]_n := [x+y]_n, 
\qquad
[x]_n \cdot [y]_n := [xy]_n.
\]

\subsection{Additive Inverse und lineare Gleichungen}
Jedes Element $[x]_n \in \mathbb{Z}/n$ besitzt ein additives Inverses $[-x]_n$
mit
\[
[x]_n + [-x]_n = [0]_n.
\]
Daher sind Gleichungen der Form
\[
a + x = b
\]
in $\mathbb{Z}/n$ für alle $a,b$ stets lösbar.

\subsection{Multiplikative Inverse}
Ein Element $[x]_n \in \mathbb{Z}/n$ besitzt genau dann ein multiplikatives
Inverses, wenn
\[
\operatorname{ggT}(n,x) = 1
\]
gilt. Insbesondere besitzt jedes Element ausser $[0]_n$ ein multiplikatives Inverses genau dann, wenn $n$ eine Primzahl ist.

\subsubsection{Multiplikatives Inverses berechnen}
Das multiplikative Inverse von $[a]_n$ kann mit dem erweiterten Euklidischen Algorithmus
berechnet werden,
indem man Zahlen $x,y \in \mathbb{Z}$ mit
\[ax + ny = 1
\]findet. Dann ist $[x]_n$ das gesuchte Inverse.

\subsection{Chinesischer Restsatz}
Sind $n_1,\dots,n_k \in \mathbb{N}_{>1}$ paarweise teilerfremd, so besitzt das Gleichungssystem simultaner
Kongruenzen
\begin{align*}
    x &\equiv_{n_1} y_1 \\
    x &\equiv_{n_2} y_2 \\
    &\vdots \\
    x &\equiv_{n_k} y_k
\end{align*}
eine eindeutige Lösung in $\mathbb{Z}/(n_1 \cdots n_k)$.

\subsubsection{Lösen simultaner Kongruenzen}
Die Lösung kann konstruiert werden, indem man die einzelnen Kongruenzen löst und die Lösungen dann kombiniert.
Man definiert
\[N = n_1 \cdots n_k,
\quad N_i = \frac{N}{n_i},
\]
und bestimmt die multiplikativen Inversen $M_i$ von $N_i$ modulo $n_i$.
Die Lösung des Gleichungssystems ist dann gegeben durch
\[x \equiv_N \sum_{i=1}^k y_i N_i M_i.\]

\subsection{Kleiner Satz von Fermat}
Ist $p$ eine Primzahl und $p \nmid a$, so gilt
\[
a^{p-1} \equiv_p 1.
\]

\subsection{Eulersche Phi-Funktion}
Für $n \in \mathbb{N}$ ist die Eulersche Phi-Funktion
\[\varphi(n) = |\{k \in \mathbb{N} \mid 1 \le k \le n, \operatorname{ggT}(k,n) = 1\}|
\]die Anzahl der positiven ganzen Zahlen bis $n$, die zu $n$ teilerfremd sind.
\begin{itemize}
    \item Ist $p$ eine Primzahl, so gilt $\varphi(p) = p-1$.
    \item Für $k \ge 1$ gilt $\varphi(p^k) = p^k - p^{k-1} = p^{k-1}(p-1)$.
    \item Ist $m,n \in \mathbb{N}$ mit $\operatorname{ggT}(m,n) = 1$, so gilt
\[
\varphi(mn) = \varphi(m) \cdot \varphi(n).
\]
    \item Für die Primfaktorzerlegung $n = p_1^{k_1} \cdots p_r^{k_r}$ gilt
\[\varphi(n) = n \left(1 - \frac{1}{p_1}\right) \cdots \left(1 - \frac{1}{p_r}\right).
\]
\end{itemize}
