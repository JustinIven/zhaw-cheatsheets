\subsection{Tupel}
Ein \(n\)-Tupel ist ein \emph{geordneter} Vektor
\[
  (a_1,\dots,a_n).
\]
Der \(i\)-te Eintrag eines Tupels \(a=(a_1,\dots,a_n)\) wird mit \(a[i]\) bezeichnet.
Zwei Tupel sind genau dann gleich, wenn sie dieselbe Länge haben und alle entsprechenden Einträge übereinstimmen:

\begin{multline*}
  (a_1,\dots,a_n)=(b_1,\dots,b_k) \iff \\
  n=k\ \land\ a_1=b_1\ \land \dots \land a_n=b_k
\end{multline*}

\subsubsection{Kartesisches Produkt}
Das kartesische Produkt \(A_1\times\cdots\times A_n\) ist die Menge aller \(n\)-Tupel, deren Einträge aus den Mengen \(A_1,\dots,A_n\) stammen.
\[
  A_1\times\cdots\times A_n:=\{(a_1,\dots,a_n)\mid a_i\in A_i\ \text{für }1\le i\le n\}.
\]

\textbf{Besonderheiten:}
\begin{itemize}
  \item Für das $n$-fache Produkt von \(A\) mit sich selbst gilt \(A^n:=A\times\cdots\times A\) (n-mal).
  \item Für ein kartesisches Produkt von der Form \(A_1\times\cdots\times A_n\) wird auch die Kurzschreibweise \(\prod_{i=1}^n A_i\) verwendet.
\end{itemize}

\textbf{Beispiele:}
\begin{align*}
  \{1\} \times \{a, b\} &= \{(1, a), (1, b)\} \\
  \mathbb{N}^2 &= \{(x, y) \mid x \in \mathbb{N} \land y \in \mathbb{N}\}
\end{align*}

\subsubsection{Projektionen}
Für eine Menge \(A\) von \(n\)-Tupeln und ist \(k \le n\) eine natürliche Zahl, definiert man die \(k\)-te Projektion:
\[
  \operatorname{pr}_k(A) := \{ x[k] \mid x \in A \}.
\]

\textbf{Insbesondere gilt:}
\[
  \operatorname{pr}_k(A_1 \times \cdots \times A_n) = A_k.
\]
\textbf{Beispiele:}
\begin{align*}
  \operatorname{pr}_1(\{1,2\} \times \{a,b\}) &= \{1, 2\} \\
  \operatorname{pr}_2(\{1,2\} \times \{a,b\}) &= \{a, b\}
\end{align*}