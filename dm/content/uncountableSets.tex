\section{Überabzählbare Mengen}
\begin{itemize}
  \item Es gibt verschiedene ``Grössen'' unendlicher Mengen; aus $|A|=\infty$ und $|B|=\infty$ folgt nicht notwendigerweise $|A|=|B|$.
  \item Es existiert eine unendliche \emph{Hierarchie} von Kardinalitäten: Mengen $A_0,A_1,A_2,\dots$ mit
    \[
      |A_0|<|A_1|<|A_2|<\dots
    \]
  \item Die Menge aller unendlichen Binärsequenzen $B=\{0,1\}^{\mathbb{N}}$ ist \emph{nicht abzählbar}.
\end{itemize}

\subsection{Sequenzen}
Eine \emph{Sequenz} in einer Menge $A$ ist eine Abbildung $s:\mathbb{N}\to A$. Die Menge aller Sequenzen in $A$ sei $A^{\mathbb{N}}$.\\
Entspricht \(s(0)=a_0, \, s(1)=a_1, \, s(2)=a_2, \, \dots\), so schreiben wir
\[  s = (a_0, a_1, a_2, \dots). \]

\subsection{Unendliche Binärsequenzen}
Eine \emph{Binärsequenz} ist eine Funktion $s:\mathbb{N}\to\{0,1\}$. Die Menge aller Binärsequenzen sei $B=\{0,1\}^{\mathbb{N}}$.

\subsection{Cantors Diagonalisierungsargument}
Für jede Abbildung $f:\mathbb{N}\to B$ konstruiere man $s\in B$ durch
\[
  s_n := 1 - f(n)_n
\]
\begin{align*}
    f(0) &= (\textcolor{red}{a^0_0}, a^0_1, a^0_2, \dots a^0_n, \dots) \\
    f(1) &= (a^1_0, \textcolor{red}{a^1_1}, a^1_2, \dots a^1_n, \dots) \\
    f(2) &= (a^2_0, a^2_1, \textcolor{red}{a^2_2}, \dots a^2_n, \dots) \\
    \vdots \\
    f(n) &= (a^n_0, a^n_1, a^n_2, \dots \textcolor{red}{a^n_n}, \dots) \\
    \vdots
\end{align*}
$s$ unterscheidet sich von jedem Bild $f(n)$ in der $\textcolor{red}{n}$-ten Stelle. Somit ist $s\notin\operatorname{im}(f)$ und es gibt keine surjektive Abbildung $\mathbb{N}\to B$. Daher ist $B$ nicht abzählbar.

\subsection{Folgerungen}
\begin{itemize}
  \item Das Intervall $[0,1)$ und damit $\mathbb{R}$ sind überabzählbar.
  \item Die Menge aller Funktionen $\mathbb{N}\to\mathbb{N}$ ist überabzählbar.
  \item Es existieren Funktionen $\mathbb{N}\to\mathbb{N}$, die nicht berechenbar sind.
\end{itemize}

\subsection{Potenzmenge und Cantors Theorem}
Für jede Menge $A$ gilt streng:
\[
  |A| < | \mathcal{P}(A) |.
\]
Begründung:
\begin{enumerate}
  \item Es existiert eine Injektion $A\hookrightarrow\mathcal{P}(A)$, $x\mapsto\{x\}$, also $|A|\le|\mathcal{P}(A)|$.
  \item Für jede Abbildung $f:A\to\mathcal{P}(A)$ betrachte die Menge
    \[
      \Delta_f := \{ a\in A \mid a\notin f(a)\}.
    \]
    $\Delta_f\in\mathcal{P}(A)$, aber $\Delta_f\notin\operatorname{im}(f)$ (diagonalisiertes Argument), also ist $f$ nicht surjektiv. Damit $|\mathcal{P}(A)|\nleq|A|$.
\end{enumerate}