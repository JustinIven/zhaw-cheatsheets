\section{Äquivalenzrelationen}
Eine Relation $\sim$ auf einer Menge $A$ heisst \emph{Äquivalenzrelation}, falls sie für alle $x,y,z\in A$ die folgenden Eigenschaften erfüllt:
\begin{itemize}
  \item \textbf{reflexiv:} \quad $x\sim x$,
  \item \textbf{symmetrisch:} \quad $x\sim y \Rightarrow y\sim x$,
  \item \textbf{transitiv:} \quad $x\sim y \land y\sim z \Rightarrow x\sim z$.
\end{itemize}

\subsection{Beispiele}
\begin{itemize}
  \item Die Gleichheitsrelation $=$ auf jeder Menge.
  \item Auf $\mathbb{Z}$: $a\equiv_n b :\Leftrightarrow n\mid(a-b)$ (Restklasse modulo $n$).
  \item Relation „sitzen in derselben Sitzreihe“ in einem Kinosaal.
\end{itemize}

\subsection{Klein. und grösst. Äquivalenzrelation}
Auf jeder Menge \(A\) existiert:
\begin{itemize}[noitemsep]
  \item die \emph{kleinste} Äquivalenzrelation: die Gleichheitsrelation \(\{(a,a)\mid a\in A\}\).
  \item die \emph{grösste} Äquivalenzrelation: das ganze \(A\times A\) (alles ist äquivalent).
\end{itemize}

\subsection{Äquivalenzklassen und Faktormenge}
Für $a\in A$ ist die \emph{Äquivalenzklasse}
\[
[a]_\sim := \{x\in A \mid x\sim a\}.
\]
Die Menge aller Äquivalenzklassen heisst \emph{Faktormenge} $A\diagup{\sim} := \{[a]_\sim \mid a\in A\}$.
Jedes Element einer Äquivalenzklasse ist ein \emph{Repräsentant} dieser Klasse.

\subsection{Wichtige Eigenschaften}
Für eine Äquivalenzrelation $\sim$ und $a,b\in A$ sind äquivalent:
\begin{enumerate}
  \item $a\sim b$,
  \item $[a]_\sim = [b]_\sim$,
  \item $[a]_\sim \cap [b]_\sim \neq \varnothing$,
  \item $a\in [b]_\sim$,
  \item $b\in [a]_\sim$.
\end{enumerate}
Daraus folgt: Zwei Äquivalenzklassen sind entweder gleich oder disjunkt.

\subsection{Beispiele in $\mathbb{R}^2$}
\begin{itemize}
  \item $(a,b)\approx(c,d) := a=c$\\ Äquivalenzklassen = vertikale Geraden.
  \item $(a,b)\simeq(c,d) := \sqrt{a^2+b^2}=\sqrt{c^2+d^2}$\\Äquivalenzklassen = Kreise um den Ursprung.
  \item Auf $\mathbb{R}^2\setminus\{(0,0)\}$: $(a,b)\sim(c,d) := \exists r\in\mathbb{R}(ra,rb)=(c,d)$\\Äquivalenzklassen = Geraden durch den Ursprung.
\end{itemize}



\subsection{Partitionen}
Eine \emph{Partition} einer Menge \(A\) ist eine Menge \(\{A_i\}_{i\in I}\) paarweise disjunkter, nichtleerer Teilmengen mit
\[
\bigcup_{i\in I} A_i = A.
\]
Die \(A_i\) nennt man auch \emph{Blöcke} der Partition.

\subsubsection{Beispiele}
\begin{itemize}
  \item Gerade und ungerade natürliche Zahlen: \(A_0=\{2n\},\; A_1=\{2n+1\}\).
  \item Einzelmengen \(\{n\}\) liefern eine feine Partition.
  \item Es gibt Partitionen von \(\mathbb{N}\) in unendlich viele unendliche Blöcke.
\end{itemize}

\subsubsection{Induzierte Partition}
Ist \(\sim\) eine Äquivalenzrelation auf \(A\), so sind die Äquivalenzklassen \([a]_\sim\) die Blöcke der Partition \(A\diagup{\sim}\).
Insbesondere sind die Klassen nichtleer und paarweise disjunkt.

\subsubsection{Induzierte Äquivalenzrelation}
Ist \(P=\{A_i\}_{i\in I}\) eine Partition von \(A\), definiert
\[
a\sim b \quad:\Leftrightarrow\quad \exists i\in I\; (a\in A_i \wedge b\in A_i)
\]
eine Äquivalenzrelation auf \(A\) mit Quotientenmenge \(A\diagup{\sim} = P\).

\subsubsection{Äquivalenzrelationen und Funktionen}
Eine Relation \(\sim\) auf \(A\) ist genau dann eine Äquivalenzrelation, wenn es eine Menge \(B\) und eine Abbildung \(f\colon A\to B\) gibt mit
\[
x\sim y \quad\Leftrightarrow\quad f(x)=f(y).
\]
(Äquivalenzklassen sind dann die Urbilder einzelner Werte von \(f\).)

