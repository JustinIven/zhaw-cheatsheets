\section{Funktionen}
Eine \emph{Funktion} \(f\) von der Menge \(A\) nach \(B\) ist eine Relation, die \emph{linksvollständig} und \emph{rechtseindeutig} ist. Man schreibt:
\[
f: A \to B,
\]
und für jedes \(x\in A\) existiert genau ein \(y\in B\) mit \(y=f(x)\).

\subsection{Schreibweise}
Oft werden Funktionen durch Angabe von Definitions- und Zielmenge sowie einer
Zuordnungsvorschrift beschrieben. Beispielsweise gilt:
\[
f = \bigl(\{(x, x^3) \mid x \in \mathbb{N}\}, \mathbb{N}, \mathbb{N}\bigr)
\]
bzw. äquivalent in der gebräuchlicheren Schreibweise:
\[
f : \mathbb{N} \to \mathbb{N}, \quad f(x) = x^3.
\]

\subsection{Injektive Funktionen}
Eine Funktion $f : A \to B$ ist \emph{injektiv}, falls die Relation \emph{linksvollständig}, \emph{rechtseindeutig} und zusätzlich \emph{linkseindeutig} ist:
\begin{align*}
  \forall x_1, x_2 &\in A (f(x_1) = f(x_2) \Rightarrow x_1 = x_2)\\
  \forall x_1, x_2 &\in A (x_1 \neq x_2 \Rightarrow f(x_1) \neq f(x_2))
\end{align*}
Jedes Element in \(A\) wird auf ein eigenes unterschiedliches Element in \(B\) abgebildet.

Notation: $f : A \hookrightarrow B$.

\subsection{Umkehrbarkeit}
Eine Funktion $f : A \to B$ ist genau dann \emph{umkehrbar}, wenn sie injektiv ist. Dann gilt:
\[
f^{-1} : \operatorname{im}(f) \to A.
\]
\[
(G^\prime_f, \operatorname{im}(f), A), \quad G^\prime_f = \{(y,x) | (x,y) \in G_f\}
\]

\subsection{Surjektivität}
Eine Funktion $f : A \to B$ ist \emph{surjektiv}, falls die Relation \emph{linksvollständig}, \emph{rechtseindeutig} und zusätzlich \emph{rechtsvollständig} ist:
\[
\operatorname{im}(f) = B
\]
Jedes Element in \(B\) wird von mindestens einem Element in \(A\) erreicht.\\
Notation: \(f:A \twoheadrightarrow B\)

\subsection{Bijektivität}
Eine Funktion $f : A \to B$ ist \emph{bijektiv}, wenn sie sowohl injektiv als auch surjektiv ist. Die Umkehrfunktion ist dann definiert durch:
\[
f^{-1} : B \to A.
\]
J
Notation: \(f : A \rightleftharpoons B\)


\subsection{Umkehrfunktion}
Für eine bijektive Funktion $f : A \rightleftharpoons B$ gilt:
\[
f^{-1} \circ f = \operatorname{id}_A, \qquad f \circ f^{-1} = \operatorname{id}_B.
\]

\subsection{Komposition}
Für $g : A \to B$ und $f : B \to C$ definiert man die \emph{Komposition}:
\[
(f \circ g)(x) = f(g(x)), \quad f \circ g : A \to C.
\]
Komposition ist \emph{assoziativ}:
\[
h \circ (g \circ f) = (h \circ g) \circ f.
\]

\subsection{Eigenschaften der Komposition}
Für Funktionen $f : A \to B$ und $g : B \to C$ gilt:
\begin{itemize}
  \item Sind $f$ und $g$ injektiv, dann ist $g \circ f$ injektiv.
  \item Sind $f$ und $g$ surjektiv, dann ist $g \circ f$ surjektiv.
  \item Sind $f$ und $g$ bijektiv, dann ist $g \circ f$ bijektiv.
\end{itemize}

