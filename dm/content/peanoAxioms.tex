\section{Peano-Axiome}
Die Peano-Axiome beschreiben die Grundstruktur der natürlichen Zahlen $\mathbb{N}$:
\begin{enumerate}[label=\textbf{Axiom \arabic*:}, leftmargin=*, itemsep=4pt]
  \item $0\in\mathbb{N}$.
  \item Zu jeder $k\in\mathbb{N}$ existiert genau ein Nachfolger $k+1\in\mathbb{N}$ (Nachfolgerfunktion $\eta : \mathbb{N} \to \mathbb{N}, \quad \eta(n)=n+1$ ).
  \item $0$ ist die einzige Zahl, die kein Nachfolger ist. \(\forall n\in\mathbb{N}\ (\forall k \in \mathbb{N}\ (n \neq k+1) \Leftrightarrow n=0)\)
  \item Die Nachfolgerfunktion \(\eta : \mathbb{N} \hookrightarrow \mathbb{N}\setminus\{0\}\) ist injektiv: $n+1=m+1\Rightarrow n=m$.
\end{enumerate}