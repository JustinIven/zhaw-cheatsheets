\documentclass[10pt,landscape]{article}
\usepackage{multicol}
\usepackage{calc}
\usepackage{ifthen}
\usepackage[landscape, a4paper, top=3mm, left=3mm, right=3mm, bottom=3mm]{geometry}
\usepackage{hyperref}
\usepackage{amsmath, amssymb}
\usepackage{enumitem}
\usepackage{tikz}
\usetikzlibrary{shapes,positioning,arrows,fit,calc,graphs,graphs.standard,intersections}
\usepackage{pgfplots}
\usepgfplotslibrary{fillbetween}
\usepackage[ngerman]{babel}



\begin{document}
% Turn off header and footer
\pagestyle{empty}
 

% Redefine section commands to use less space
\makeatletter
\renewcommand{\section}{\@startsection{section}{1}{0mm}%
                                {-1ex plus -.5ex minus -.2ex}%
                                {0.5ex plus .2ex}%x
                                {\color{blue}\normalfont\large\bfseries}}
\renewcommand{\subsection}{\@startsection{subsection}{2}{0mm}%
                                {-1explus -.5ex minus -.2ex}%
                                {0.5ex plus .2ex}%
                                {\normalfont\normalsize\bfseries}}
\renewcommand{\subsubsection}{\@startsection{subsubsection}{3}{0mm}%
                                {-1ex plus -.5ex minus -.2ex}%
                                {1ex plus .2ex}%
                                {\normalfont\small\bfseries}}
\makeatother

% Remove indentation in itemize/enumerate and set left margin to zero
\setlist[itemize]{leftmargin=*}
\setlist[enumerate]{leftmargin=*}

% Don't print section numbers
\setcounter{secnumdepth}{0}

% Set font size and alignment for the whole document
\raggedright
\footnotesize

% Set column separation
% \setlength{\columnseprule}{1pt}
% \def\columnseprulecolor{\color{blue}}

\begin{multicols*}{4}

\begin{center}
     \Large{\textbf{Diskrete Mathematik}} \\
\end{center}

\section{Zahlenmengen}
\begin{tabular}{@{}ll@{}}
\(\mathbb{N}\) & natürliche Zahlen \\
\(\mathbb{N}_0\) & natürliche Zahlen mit 0 \\
\(\mathbb{Z}\) & ganze Zahlen \\
\(\mathbb{Q}\) & rationale Zahlen \\
\(\mathbb{R}\) & reelle Zahlen \\
\(\mathbb{C}\) & komplexe Zahlen \\
\end{tabular}

\section{Aussagenlogik}
\begin{tabular}{@{} p{1.2cm}|p{4.5cm}| @{}}
Aussage & Ein Satz, der entweder wahr (w) oder falsch (f) ist. \\
Prädikat & Eine Aussage mit Variablen. {\it n}-stellige Prädikate. \\
\end{tabular}

\subsection{Grundidee}
Aus gegebenen Prädikaten/Aussagen lassen sich durch Junktoren neue Aussagen bilden. (z.\,B. Kombinationen mit \(\land,\lor,\lnot,\Rightarrow,\Leftrightarrow\)).

\subsection{Definitionen}
\begin{itemize}
  \item \textbf{Negation:} \(\lnot A\) ist genau dann wahr, wenn \(A\) falsch ist. (Doppelte Negation: \(A\Leftrightarrow\lnot\lnot A\).)
  \item \textbf{Konjunktion:} \(A\land B\) ist wahr genau dann, wenn \(A\) und \(B\) wahr sind. (assoziativ, kommutativ, idempotent)
  \item \textbf{Disjunktion:} \(A\lor B\) ist wahr, wenn mindestens eine der Aussagen wahr ist. (assoziativ, kommutativ, idempotent)
  \item \textbf{Implikation:} \(A\Rightarrow B\) ist äquivalent zu \(\lnot A\lor B\). (Kontraposition: \(A\Rightarrow B \Leftrightarrow \lnot B\Rightarrow\lnot A\).)
  \item \textbf{Äquivalenz:} \(A\Leftrightarrow B\) genau dann, wenn \(A\Rightarrow B \land B\Rightarrow A\).
\end{itemize}

\subsection{Wichtige Regeln}
\begin{itemize}
  \item \textbf{De Morgan:}\\ 
    \(\lnot(A\land B)\Leftrightarrow \lnot A\lor\lnot B\) 
    \(\lnot(A\lor B)\Leftrightarrow\lnot A\land\lnot B\)
  \item \textbf{Distributivität:}
    \(A\land(B\lor C)\Leftrightarrow (A\land B)\lor(A\land C)\)
    \(A\lor(B\land C)\Leftrightarrow (A\lor B)\land(A\lor C)\)
  \item \textbf{Syntaktische Bindung:} \(\lnot\) bindet stärker als \(\land,\lor\); diese binden stärker als \(\Rightarrow,\Leftrightarrow\).
  \item \textbf{Modus Ponens:} Aus \(A \land (A\Rightarrow B\)) folgt \(B\).
  \item \textbf{Transitivität:} Aus \((A\Rightarrow B) \land (B\Rightarrow C)\) folgt \(A\Rightarrow C\).
\end{itemize}

\subsection{Hinweis zur Redundanz}
Jeder Ausdruck mit den Junktoren $\lnot,\land,\lor,\Rightarrow$ lässt sich ausschliesslich mit \(\lnot\) und \(\lor\) darstellen. z.B.
$$A\land B \Leftrightarrow \lnot(\lnot A\lor\lnot B)$$

\section{Quantoren}
Quantoren dienen zur Formalisierung von Aussagen wie:
\begin{itemize}
  \item $\forall x\,A(x)$: \emph{Für alle $x$ gilt $A(x)$}
  \item $\exists x\,A(x)$: \emph{Es existiert ein $x$ mit $A(x)$}
\end{itemize}

Mehrere gleichartige Quantoren:
$$\forall x,y\;A(x,y) \quad\text{statt}\quad \forall x\,\forall y\;A(x,y)$$


\subsection{Eingeschränkte Quantoren}
$$\forall x \in M\,A(x): \text{Für alle }x\in M \text{ gilt }A(x)$$
$$\exists x \in M\,A(x): \text{Es gibt }x\in M \text{ mit }A(x)$$

Auch möglich mit Relationen:
$$\forall x < y\,A(x) \quad \text{oder} \quad \exists x \le y\,A(x)$$

\subsection{Als Junktoren}
Für endliche Mengen $M = \{x_1, \dots, x_n\}$ gilt:
$$\forall x \in M\,A(x) \Leftrightarrow A(x_1)\land \dots \land A(x_n)$$
$$\exists x \in M\,A(x) \Leftrightarrow A(x_1)\lor \dots \lor A(x_n)$$

\subsection{Als Makros}
$$\exists x \in M\,A(x) \Leftrightarrow \exists x\,(x \in M \land A(x))$$
$$\forall x \in M\,A(x) \Leftrightarrow \forall x\,(x \in M \Rightarrow A(x))$$

\subsection{Zusammenhang mit Junktoren}
$$\neg \forall x\,A(x) \Leftrightarrow \exists x\,\neg A(x)
\quad\text{und}\quad
\neg \exists x\,A(x) \Leftrightarrow \forall x\,\neg A(x)$$
$$\forall x\,(A(x)\land B(x)) \Leftrightarrow (\forall x\,A(x)) \land (\forall x\,B(x))$$
$$\exists x\,(A(x)\lor B(x)) \Leftrightarrow (\exists x\,A(x)) \lor (\exists x\,B(x))$$

\subsection{Leere Quantoren}
Wenn $x$ in $B$ nicht vorkommt:
$$\forall x\,B \Leftrightarrow B, \quad \exists x\,B \Leftrightarrow B$$

\section{Mengen}
\begin{itemize}
  \item \textbf{Menge / Element:} Eine Menge fasst mathematische Objekte (Elemente) zu einem Ganzen zusammen. Für Menge \(X\) und Element \(y\) gilt \(y\in X\) bzw.\ \(y\notin X\).
  \item \textbf{Aufzählende Schreibweise:} \(\{x_1,\dots,x_n\}\) bezeichnet die Menge, die genau die genannten Elemente enthält. Die leere Menge heisst \(\varnothing\).
  \item \textbf{Extensionalitätsprinzip:} Zwei Mengen sind genau dann gleich, wenn sie dieselben Elemente haben:
  \[ A=B \iff \forall x\,(x\in A \Leftrightarrow x\in B). \]
  \item \textbf{Teilmenge:} \(A\subseteq B\) genau dann, wenn \(\forall x\,(x\in A \Rightarrow x\in B)\). Ist \(A\subseteq B\) und \(A\neq B\), so ist \(A\) eine \emph{echte} Teilmenge, geschrieben \(A\subset B\).
  \item \textbf{Folgerungen:} Mengen sind ungeordnet; Mehrfachaufzählung desselben Elements ändert die Menge nicht. Für jede Menge \(A\) gilt \(\varnothing\subseteq A\).
\end{itemize}

\subsection{Eindeutigkeit der leeren Menge}
Seien \(e_1,e_2\) leere Mengen. Dann ist für alle \(x\) die Aussage \(x\in e_1\) falsch, also ist die Implikation \(x\in e_1\Rightarrow x\in e_2\) wahr; somit \(e_1\subseteq e_2\). Analog \(e_2\subseteq e_1\). Nach Extensionalität folgt \(e_1=e_2\).


\subsection{Aussonderungsprinzip}
Ist \(A\) eine Menge und \(E(x)\) eine Eigenschaft, dann gilt:
\[
\{x \in A \mid E(x)\} = \text{Menge aller } x \in A \text{ mit } E(x).
\]
\[
a \in \{x \in A \mid E(x)\} \iff a \in A \land E(a)
\]

\textbf{Beispiele:}
\begin{itemize}
  \item Gerade Zahlen: \(\{x \in \mathbb{N} \mid \exists y \in \mathbb{N} (x = 2y)\}\)
  \item Zahlen \(> 17\): \(\{x \in \mathbb{N} \mid x > 17\}\)
  \item Alle ausser \(22\): \(\{x \in \mathbb{N} \mid x \neq 22\}\)
\end{itemize}

\subsection{Ersetzungsprinzip}
Ist \(A\) eine Menge und \(t(x)\) ein Ausdruck, so gilt:
\[ \{t(x) \mid x \in A\} = \text{Menge aller Werte von } t(x) \text{ mit } x \in A. \]
\[ a \in \{t(x) \mid x \in A\} \iff \exists x \in A (a = t(x)) \]

\textbf{Beispiele:}
\begin{itemize}
  \item Quadratzahlen: \(\{x^2 \mid x \in \mathbb{N}\}\)
  \item Ungerade Zahlen: \(\{2x + 1 \mid x \in \mathbb{N}\}\)
  \item Rationale Zahlen: \(\left\{\frac{a}{b} \mid a,b \in \mathbb{Z},\, b \neq 0\right\}\)
  \item Anfangsabschnitte von \(\mathbb{N}\): \(\{\{x \in \mathbb{N} \mid x < y\} \mid y \in \mathbb{N}\}\)
\end{itemize}




\subsection{Vereinigung}
Die Vereinigung von zwei Mengen beinhaltet genau die Elemente, die in mindestens einer der beiden Mengen enthalten sind:
\[
A\cup B:=\{x\mid x\in A\vee x\in B\}.
\]

\subsection{Schnitt}
Die Schnittmenge von zwei Mengen beinhaltet genau die Elemente, die in beiden Mengen enthalten sind:
\[
A\cap B:=\{x\mid x\in A\land x\in B\}.
\]

\subsection{Allgemeine Vereinigung / Schnitt}
Sei \(I\) eine beliebige Indexmenge (z.\,B. \(I = \{1,2,\dots,n\}\) oder \(I = \mathbb{N}\)).  
Für jedes \(i \in I\) sei \(A_i\) eine Menge.

\subsubsection{Allgemeine Vereinigung}
\(x\) gehört zur Vereinigung genau dann, wenn es in \emph{mindestens einer} der Mengen \(A_i\) enthalten ist.
\[
\bigcup_{i \in I} A_i := \{\, x \mid \exists i \in I : x \in A_i \,\}.
\]


\subsubsection{Allgemeiner Schnitt}
\(x\) gehört zum Schnitt genau dann, wenn es in \emph{allen} Mengen \(A_i\) enthalten ist.
\[
\bigcap_{i \in I} A_i := \{\, x \mid \forall i \in I : x \in A_i \,\}.
\]

\subsection{Differenz}
Die Differenz von zwei Mengen beinhaltet genau die Elemente, die in der ersten Menge, aber nicht in der zweiten Menge enthalten sind:
\[
A\setminus B:=\{x\in A\mid x\notin B\}.
\]

\subsection*{Disjunkte Mengen}
Zwei Mengen \(A\) und \(B\) heissen disjunkt, wenn sie keine gemeinsamen Elemente besitzen.
\[
A \cap B = \varnothing.
\]

\subsection{Paarweise disjunkt}
Eine Familie von Mengen \((A_i)_{i \in I}\) heisst paarweise disjunkt, wenn keine zwei verschiedenen Mengen ein gemeinsames Element haben. Es gilt:
\[
\forall i,j \in I \; (i \neq j \Rightarrow A_i \cap A_j = \varnothing).
\]

\subsection{Wichtige Eigenschaften}
Für beliebige Mengen \(A,B,C\) gelten:
\begin{itemize}
  \item Idempotenz: \(A\cup A=A,\; A\cap A=A\).
  \item Kommutativität: \(A\cup B=B\cup A,\; A\cap B=B\cap A\).
  \item Assoziativität: \(A\cup(B\cup C)=(A\cup B)\cup C\) und analog für \(\cap\).
  \item Teilmengen: \(A\subseteq A\cup B\) und \(A\cap B\subseteq A\).
  \item Distributivität:
  \[
  A\cup(B\cap C)=(A\cup B)\cap(A\cup C),
  \]
  \[
  A\cap(B\cup C)=(A\cap B)\cup(A\cap C).
  \]
  \item De Morgansche Regeln:
  \[
  C\setminus(A\cap B)=(C\setminus A)\cup(C\setminus B),
  \]
  \[
  C\setminus(A\cup B)=(C\setminus A)\cap(C\setminus B).
  \]
\end{itemize}

\subsection{Venn-Diagramm}
% Set A and B
\begin{tikzpicture}[scale=.45]
    \begin{scope}
        \clip \firstcircle;
        \fill[filled] \secondcircle;
    \end{scope}
    \draw[outline] \firstcircle node {$A$};
    \draw[outline] \secondcircle node {$B$};
    \node[anchor=south] at (current bounding box.north) {$A \cap B$};
\end{tikzpicture}
%Set A or B but not (A and B) also known a A xor B
\begin{tikzpicture}[scale=.45]
    \draw[filled, even odd rule] \firstcircle node {$A$}
                                 \secondcircle node{$B$};
    \node[anchor=south] at (current bounding box.north) {$\overline{A \cap B}$};
\end{tikzpicture}
% Set A or B
\begin{tikzpicture}[scale=.45]
    \draw[filled] \firstcircle node {$A$}
                  \secondcircle node {$B$};
    \node[anchor=south] at (current bounding box.north) {$A \cup B$};
\end{tikzpicture}
% Set A but not B
\begin{tikzpicture}[scale=.45]
    \begin{scope}
        \clip \firstcircle;
        \draw[filled, even odd rule] \firstcircle node {$A$}
                                     \secondcircle;
    \end{scope}
    \draw[outline] \firstcircle
                   \secondcircle node {$B$};
    \node[anchor=south] at (current bounding box.north) {$A\setminus B$};
\end{tikzpicture}

\subsection{Potenzmenge}
Für eine Menge \(A\) bezeichnet die Potenzmenge \( \mathcal{P}(A) \) die Menge aller Teilmengen von \(A\):
\[
\mathcal{P}(A):=\{X \mid X\subseteq A\}
\]

\textbf{Beispiele:}
\begin{align*}
\mathcal{P}(\{1,2\}) &= \{\varnothing,\{1\},\{2\},\{1,2\}\},\\
\mathcal{P}(\varnothing) &= \{\varnothing\},\\
\mathcal{P}(\{\{a\}\}) &= \{\varnothing,\{\{a\}\}\}.
\end{align*}

\textbf{Eigenschaften:}
\begin{itemize}
  \item \(A\in\mathcal{P}(A)\) und \(\varnothing\in\mathcal{P}(A)\).
  \item Aus \(A\subseteq B\) folgt \(\mathcal{P}(A)\subseteq\mathcal{P}(B)\).
  \item Für die leere Menge gilt \(\mathcal{P}(\varnothing)=\{\varnothing\}\neq\varnothing\).
  \item \(\mathcal{P}(A\cap B)=\mathcal{P}(A)\cap\mathcal{P}(B)\)
  \item \(\mathcal{P}(A\cup B)\supseteq \mathcal{P}(A)\cup\mathcal{P}(B)\)
\end{itemize}

\section{Relationen und Funktionen}
\subsection{Tupel}
Ein \(n\)-Tupel ist ein \emph{geordneter} Vektor
\[
  (a_1,\dots,a_n).
\]
Der \(i\)-te Eintrag eines Tupels \(a=(a_1,\dots,a_n)\) wird mit \(a[i]\) bezeichnet.
Zwei Tupel sind genau dann gleich, wenn sie dieselbe Länge haben und alle entsprechenden Einträge übereinstimmen:

\begin{multline*}
  (a_1,\dots,a_n)=(b_1,\dots,b_k) \iff \\
  n=k\ \land\ a_1=b_1\ \land \dots \land a_n=b_k
\end{multline*}

\subsubsection{Kartesisches Produkt}
Das kartesische Produkt \(A_1\times\cdots\times A_n\) ist die Menge aller \(n\)-Tupel, deren Einträge aus den Mengen \(A_1,\dots,A_n\) stammen.
\[
  A_1\times\cdots\times A_n:=\{(a_1,\dots,a_n)\mid a_i\in A_i\ \text{für }1\le i\le n\}.
\]

\textbf{Besonderheiten:}
\begin{itemize}
  \item Für das $n$-fache Produkt von \(A\) mit sich selbst gilt \(A^n:=A\times\cdots\times A\) (n-mal).
  \item Für ein kartesisches Produkt von der Form \(A_1\times\cdots\times A_n\) wird auch die Kurzschreibweise \(\prod_{i=1}^n A_i\) verwendet.
\end{itemize}

\textbf{Beispiele:}
\begin{align*}
  \{1\} \times \{a, b\} &= \{(1, a), (1, b)\} \\
  \mathbb{N}^2 &= \{(x, y) \mid x \in \mathbb{N} \land y \in \mathbb{N}\}
\end{align*}

\subsubsection{Projektionen}
Für eine Menge \(A\) von \(n\)-Tupeln und ist \(k \le n\) eine natürliche Zahl, definiert man die \(k\)-te Projektion:
\[
  \operatorname{pr}_k(A) := \{ x[k] \mid x \in A \}.
\]

\textbf{Insbesondere gilt:}
\[
  \operatorname{pr}_k(A_1 \times \cdots \times A_n) = A_k.
\]
\textbf{Beispiele:}
\begin{align*}
  \operatorname{pr}_1(\{1,2\} \times \{a,b\}) &= \{1, 2\} \\
  \operatorname{pr}_2(\{1,2\} \times \{a,b\}) &= \{a, b\}
\end{align*}
\section{Relationen}
Eine \emph{Relation} von \(A\) nach \(B\) ist ein Tripel
\[
R=(G,A,B)
\]
wobei \(A\) die Quellmenge, \(B\) die Zielmenge und \(G\subseteq A\times B\) der \emph{Graph} von \(R\) ist. Ist \(A=B\), so heisst \(R\) \emph{homogen} auf \(A\).

\subsection{Notation}
Sei \(R=(G,A,B)\) eine Relation von \(A\) nach \(B\).
\begin{itemize}
  \item Ist \(G\) der Graph von \(R\), so schreibt man \(G_R\)
  \item Ist \((x,y)\in G\), dann schreibt man \(xRy\) (\(x\) steht in Relation zu \(y\) bezüglich \(R\)).
  \item Sind \(A\) und \(B\) Teilmengen von \(\mathbb{R}\), so kann man \(R\) auch als Menge von Punkten in der Ebene darstellen: \(\{(x,y)\mid xRy\}\).\\
    % 1. x^2 = y^2
    \begin{tikzpicture}[scale=.45]
      \begin{axis}[axis lines=middle, xmin=-2, xmax=2, ymin=-2, ymax=2, width=5cm, height=5cm]
        \addplot[domain=-3:3, samples=100, blue!30] {x};
        \addplot[domain=-3:3, samples=100, blue!30] {-x};
      \end{axis}
      \node[anchor=south] at (current bounding box.north) {\tiny{$xRy: \Leftrightarrow x^2 = y^2$}};
    \end{tikzpicture}
    % 2. x^2 + y^2 = 1
    \begin{tikzpicture}[scale=.45]
      \begin{axis}[axis lines=middle, xmin=-2, xmax=2, ymin=-2, ymax=2, width=5cm, height=5cm]
        \addplot[domain=0:360, samples=200, blue!30] ({cos(x)}, {sin(x)});
      \end{axis}
      \node[anchor=south] at (current bounding box.north) {\tiny{$xRy: \Leftrightarrow x^2 + y^2 = 1$}};
    \end{tikzpicture}
  \item Als gerichteter Graph: Elemente von \(A\) und \(B\) als Knoten; für jedes \((x,y)\in G\) ein Pfeil \(x\to y\).\\
    \begin{tikzpicture}[main/.style = {draw, circle}] 
      \node[main] (1) {$1$}; 
      \node[main] (2) [right=0.5cm of 1] {$2$}; 
      \node[main] (3) [below=0.5cm of 1] {$3$}; 
      \node[main] (4) [right=0.5cm of 3] {$4$};
      \draw[->] (1) to (2);
      \draw[->] (1) to (3);
      \draw[->] (1) to (4);
      \draw[->] (2) to (4);
      \draw[->] (1) to [out=90,in=140,looseness=4] (1);
      \draw[->] (2) to [out=90,in=140,looseness=4] (2);
      \draw[->] (3) to [out=180,in=230,looseness=4] (3);
      \draw[->] (4) to [out=180,in=230,looseness=4] (4);
      \node[anchor=south] at (current bounding box.north) {\tiny{$xRy: \Leftrightarrow x \text{ teilt } y$}};
    \end{tikzpicture}
    \begin{tikzpicture}[main/.style = {draw, circle}] 
      \node[main] (1) {$1$}; 
      \node[main] (2) [right=0.5cm of 1] {$2$}; 
      \node[main] (3) [below=0.5cm of 1] {$3$}; 
      \node[main] (4) [right=0.5cm of 3] {$4$};
      \draw[<->] (1) to (3);
      \draw[<->] (2) to (4);
      \draw[->] (1) to [out=90,in=140,looseness=4] (1);
      \draw[->] (2) to [out=90,in=140,looseness=4] (2);
      \draw[->] (3) to [out=180,in=230,looseness=4] (3);
      \draw[->] (4) to [out=180,in=230,looseness=4] (4);
      \node[anchor=south] at (current bounding box.north) {\tiny{$xRy: \Leftrightarrow x+y \text{ ist gerade}$}};
    \end{tikzpicture} 
\end{itemize}


\subsection{Domäne und Bild}
Die Domäne und das Bild einer Relation geben an, welche Elemente der Quell- bzw. Zielmenge tatsächlich in der Relation vorkommen.
\begin{align*}
  \operatorname{dom}(R)&:=\operatorname{pr}_1(G_R)=\{a\in A\mid \exists b\in B(aRb)\}\\
  \operatorname{im}(R)&:=\operatorname{pr}_2(G_R)=\{b\in B\mid \exists a\in A(aRb)\}
\end{align*}
Im gerichteten Graphen entsprechen die Elemente der Domäne den Knoten mit ausgehenden Kanten, die des Bildes den Knoten mit eingehenden Kanten.


\subsection{Klassifizierungen}
Sei \(R\subseteq A\times A\) eine (homogene) Relation auf \(A\).

\subsubsection{Reflexivität}
Eine Relation \(R\) heisst \emph{reflexiv}, wenn jedes Element in Relation zu sich selbst steht:
\[\forall x\in A(xRx)\]
\begin{itemize}
    \item \(\{(a,a)\mid a\in A\}\subseteq R\).
    \item Im gerichteten Graphen hat jeder Knoten eine Kante zu sich selbst. Für jeden Wert \(x\in A\) gilt:
      \begin{tikzpicture}[main/.style = {draw, circle}] 
        \node[main] (x) {$x$};
        \draw[->] (x) to [out=90,in=140,looseness=4] (x);
      \end{tikzpicture}
    \item In der Koordinatendarstellung enthält \(R\) die Winkelhalbierende \(y=x\).
\end{itemize}

\subsubsection{Symmetrie}
Eine Relation \(R\) heisst \emph{symmetrisch}, wenn für alle \(x,y\in A\) gilt:
\[
  \forall x,y\;(xRy\Rightarrow yRx).
\]
\begin{itemize}
  \item Zu jedem Pfeil im gerichteten Graph existiert der umgekehrte Pfeil. Für alle \(x,y\in A\) gilt:\\
    \begin{tikzpicture}[main/.style = {draw, circle}] 
      \node[main] (x) {$x$};
      \node[main] (y) [right=0.5cm of x] {$y$};
      \draw[<->] (x) to (y);
    \end{tikzpicture}
  \item Symmetrie spiegelt die Koordinatendarstellung an der Geraden \(y=x\).
\end{itemize}

\subsubsection{Antisymmetrie}
Eine Relation \(R\) heisst \emph{antisymmetrisch}, wenn für
alle \(x,y\in A\) gilt:
\[
    \forall x,y\;(xRy\land yRx\Rightarrow x=y).
\]
\begin{itemize}
  \item Es gibt keine zwei verschiedenen Knoten, die wechselseitig verbunden sind.
    Für alle \(x,y\in A, x\neq y\) gilt:\\
    \begin{tikzpicture}[main/.style = {draw, circle}] 
      \node[main] (x) {$x$};
      \node[main] (y) [right=0.5cm of x] {$y$};
      \draw[->] (x) to (y);
    \end{tikzpicture}
\end{itemize}

\subsubsection{Transitivität}
Eine Relation \(R\) heisst \emph{transitiv}, wenn für jeden endlichen Pfad ein direkter Pfeil existiert. Für alle \(x,y,z\in A\) gilt:
\[
  \forall x,y,z\;(xRy\land yRz\Rightarrow xRz).
\]
\begin{itemize}
    \item Im gerichteten Graphen: Aus \(x\to y\) und \(y\to z\) folgt \(x\to z\). Für alle \(x,y,z\in A\) gilt:\\
      \begin{tikzpicture}[main/.style = {draw, circle}] 
        \node[main] (x) {$x$};
        \node[main] (y) [right=0.5cm of x] {$y$};
        \node[main] (z) [right=0.5cm of y] {$z$};
        \draw[->] (x) to (y);
        \draw[->] (y) to (z);
        \draw[->, dashed] (x) to [bend left] (z);
      \end{tikzpicture}
\end{itemize}

\subsubsection{Totalität und Eindeutigkeit}
Sei \(R\subseteq A\times B\) eine Relation von \(A\) nach \(B\) mit
\begin{itemize}
  \item \textbf{Linksvollständig / linkstotal:} \(\mathrm{dom}(R)=A\) (jedes Element in \(A\) hat min. eine \emph{ausgehende} Kante).
  \item \textbf{Rechtsvollständig / rechtstotal:} \(\mathrm{im}(R)=B\) (jedes Element in \(B\) hat min. eine \emph{eingehende} Kante).
  \item \textbf{Linkseindeutig:} \(\forall x_1,x_2,y\;(x_1Ry\land x_2Ry\Rightarrow x_1=x_2)\) (jedes Element in \(B\) hat max. eine \emph{eingehende} Kante).
  \item \textbf{Rechtseindeutig:} \(\forall x,y_1,y_2\;(xRy_1\land xRy_2\Rightarrow y_1=y_2)\) (jedes Element in \(A\) hat max. eine \emph{ausgehende} Kante).
\end{itemize}

\subsection{Inverse Relationen}
Für eine Relation $R = (G, A, B)$ ist die \emph{inverse Relation} definiert durch
\[
R^{-1} = (G', B, A), \quad G' := \{ (y,x) \mid (x,y) \in G \}.
\]
\textbf{Eigenschaften:}  
\begin{itemize}
    \item $(R^{-1})^{-1} = R$
    \item $R$ ist linksvollständig $\Leftrightarrow R^{-1}$ ist rechtsvollständig
    \item $R$ ist linkseindeutig $\Leftrightarrow R^{-1}$ ist rechtseindeutig
    \item Für jede symmetrische Relation $R$ gilt $R = R^{-1}$
\end{itemize}
\subsection{Klassifizierungen}

Sei \(R\subseteq A\times A\) eine (homogene) Relation auf \(A\).

\subsubsection{Reflexivität}
Eine Relation \(R\) heisst \emph{reflexiv}, wenn jedes Element in Relation zu sich selbst steht:
\[\forall x\in A(xRx)\]
\begin{itemize}
    \item \(\{(a,a)\mid a\in A\}\subseteq R\).
    \item Im gerichteten Graphen hat jeder Knoten eine Kante zu sich selbst. Für jeden Wert \(x\in A\) gilt:
      \begin{tikzpicture}[main/.style = {draw, circle}] 
        \node[main] (x) {$x$};
        \draw[->] (x) to [out=90,in=140,looseness=4] (x);
      \end{tikzpicture}
    \item In der Koordinatendarstellung enthält \(R\) die Winkelhalbierende \(y=x\).
\end{itemize}

\subsubsection{Symmetrie}
Eine Relation \(R\) heisst \emph{symmetrisch}, wenn für alle \(x,y\in A\) gilt:
\[
  \forall x,y\;(xRy\Rightarrow yRx).
\]
\begin{itemize}
  \item Zu jedem Pfeil im gerichteten Graph existiert der umgekehrte Pfeil. Für alle \(x,y\in A\) gilt:\\
    \begin{tikzpicture}[main/.style = {draw, circle}] 
      \node[main] (x) {$x$};
      \node[main] (y) [right=0.5cm of x] {$y$};
      \draw[<->] (x) to (y);
    \end{tikzpicture}
  \item Symmetrie spiegelt die Koordinatendarstellung an der Geraden \(y=x\).
\end{itemize}

\subsubsection{Antisymmetrie}
Eine Relation \(R\) heisst \emph{antisymmetrisch}, wenn für
alle \(x,y\in A\) gilt:
\[
    \forall x,y\;(xRy\land yRx\Rightarrow x=y).
\]
\begin{itemize}
  \item Es gibt keine zwei verschiedenen Knoten, die wechselseitig verbunden sind.
    Für alle \(x,y\in A, x\neq y\) gilt:\\
    \begin{tikzpicture}[main/.style = {draw, circle}] 
      \node[main] (x) {$x$};
      \node[main] (y) [right=0.5cm of x] {$y$};
      \draw[->] (x) to (y);
    \end{tikzpicture}
\end{itemize}

\subsubsection{Transitivität}
Eine Relation \(R\) heisst \emph{transitiv}, wenn für jeden endlichen Pfad ein direkter Pfeil existiert. Für alle \(x,y,z\in A\) gilt:
\[
  \forall x,y,z\;(xRy\land yRz\Rightarrow xRz).
\]
\begin{itemize}
    \item Im gerichteten Graphen: Aus \(x\to y\) und \(y\to z\) folgt \(x\to z\). Für alle \(x,y,z\in A\) gilt:\\
      \begin{tikzpicture}[main/.style = {draw, circle}] 
        \node[main] (x) {$x$};
        \node[main] (y) [right=0.5cm of x] {$y$};
        \node[main] (z) [right=0.5cm of y] {$z$};
        \draw[->] (x) to (y);
        \draw[->] (y) to (z);
        \draw[->, dashed] (x) to [bend left] (z);
      \end{tikzpicture}
\end{itemize}

\subsubsection{Totalität und Eindeutigkeit}
Sei \(R\subseteq A\times B\) eine Relation von \(A\) nach \(B\) mit
\begin{itemize}
  \item \textbf{Linksvollständig / linkstotal:} \(\mathrm{dom}(R)=A\) (jedes Element in \(A\) hat min. eine \emph{ausgehende} Kante).
  \item \textbf{Rechtsvollständig / rechtstotal:} \(\mathrm{im}(R)=B\) (jedes Element in \(B\) hat min. eine \emph{eingehende} Kante).
  \item \textbf{Linkseindeutig:} \(\forall x_1,x_2,y\;(x_1Ry\land x_2Ry\Rightarrow x_1=x_2)\) (jedes Element in \(B\) hat max. eine \emph{eingehende} Kante).
  \item \textbf{Rechtseindeutig:} \(\forall x,y_1,y_2\;(xRy_1\land xRy_2\Rightarrow y_1=y_2)\) (jedes Element in \(A\) hat max. eine \emph{ausgehende} Kante).
\end{itemize}

\subsection{Inverse Relationen}
Für eine Relation $R = (G, A, B)$ ist die \emph{inverse Relation} definiert durch
\[
R^{-1} = (G', B, A), \quad G' := \{ (y,x) \mid (x,y) \in G \}.
\]
\textbf{Eigenschaften:}  
\begin{itemize}
    \item $(R^{-1})^{-1} = R$
    \item $R$ ist linksvollständig $\Leftrightarrow R^{-1}$ ist rechtsvollständig
    \item $R$ ist linkseindeutig $\Leftrightarrow R^{-1}$ ist rechtseindeutig
    \item Für jede symmetrische Relation $R$ gilt $R = R^{-1}$
\end{itemize}
\section{Funktionen}
Eine \emph{Funktion} \(f\) von der Menge \(A\) nach \(B\) ist eine Relation, die \emph{linksvollständig} und \emph{rechtseindeutig} ist. Man schreibt:
\[
f: A \to B,
\]
und für jedes \(x\in A\) existiert genau ein \(y\in B\) mit \(y=f(x)\).

\subsection{Schreibweise}
Oft werden Funktionen durch Angabe von Definitions- und Zielmenge sowie einer
Zuordnungsvorschrift beschrieben. Beispielsweise gilt:
\[
f = \bigl(\{(x, x^3) \mid x \in \mathbb{N}\}, \mathbb{N}, \mathbb{N}\bigr)
\]
bzw. äquivalent in der gebräuchlicheren Schreibweise:
\[
f : \mathbb{N} \to \mathbb{N}, \quad f(x) = x^3.
\]

\subsection{Injektive Funktionen}
Eine Funktion $f : A \to B$ ist \emph{injektiv}, falls die Relation \emph{linksvollständig}, \emph{rechtseindeutig} und zusätzlich \emph{linkseindeutig} ist:
\begin{align*}
  \forall x_1, x_2 &\in A (f(x_1) = f(x_2) \Rightarrow x_1 = x_2)\\
  \forall x_1, x_2 &\in A (x_1 \neq x_2 \Rightarrow f(x_1) \neq f(x_2))
\end{align*}
Jedes Element in \(A\) wird auf ein eigenes unterschiedliches Element in \(B\) abgebildet.

Notation: $f : A \hookrightarrow B$.

\subsection{Umkehrbarkeit}
Eine Funktion $f : A \to B$ ist genau dann \emph{umkehrbar}, wenn sie injektiv ist. Dann gilt:
\[
f^{-1} : \operatorname{im}(f) \to A.
\]
\[
(G^\prime_f, \operatorname{im}(f), A), \quad G^\prime_f = \{(y,x) | (x,y) \in G_f\}
\]

\subsection{Surjektivität}
Eine Funktion $f : A \to B$ ist \emph{surjektiv}, falls die Relation \emph{linksvollständig}, \emph{rechtseindeutig} und zusätzlich \emph{rechtsvollständig} ist:
\[
\operatorname{im}(f) = B
\]
Jedes Element in \(B\) wird von mindestens einem Element in \(A\) erreicht.\\
Notation: \(f:A \twoheadrightarrow B\)

\subsection{Bijektivität}
Eine Funktion $f : A \to B$ ist \emph{bijektiv}, wenn sie sowohl injektiv als auch surjektiv ist. Die Umkehrfunktion ist dann definiert durch:
\[
f^{-1} : B \to A.
\]
J
Notation: \(f : A \rightleftharpoons B\)


\subsection{Umkehrfunktion}
Für eine bijektive Funktion $f : A \rightleftharpoons B$ gilt:
\[
f^{-1} \circ f = \operatorname{id}_A, \qquad f \circ f^{-1} = \operatorname{id}_B.
\]

\subsection{Komposition}
Für $g : A \to B$ und $f : B \to C$ definiert man die \emph{Komposition}:
\[
(f \circ g)(x) = f(g(x)), \quad f \circ g : A \to C.
\]
Komposition ist \emph{assoziativ}:
\[
h \circ (g \circ f) = (h \circ g) \circ f.
\]

\subsection{Eigenschaften der Komposition}
Für Funktionen $f : A \to B$ und $g : B \to C$ gilt:
\begin{itemize}
  \item Sind $f$ und $g$ injektiv, dann ist $g \circ f$ injektiv.
  \item Sind $f$ und $g$ surjektiv, dann ist $g \circ f$ surjektiv.
  \item Sind $f$ und $g$ bijektiv, dann ist $g \circ f$ bijektiv.
\end{itemize}


\section{Äquivalenzrelationen}
Eine Relation $\sim$ auf einer Menge $A$ heisst \emph{Äquivalenzrelation}, falls sie für alle $x,y,z\in A$ die folgenden Eigenschaften erfüllt:
\begin{itemize}
  \item \textbf{reflexiv:} \quad $x\sim x$,
  \item \textbf{symmetrisch:} \quad $x\sim y \Rightarrow y\sim x$,
  \item \textbf{transitiv:} \quad $x\sim y \land y\sim z \Rightarrow x\sim z$.
\end{itemize}

\subsection{Beispiele}
\begin{itemize}
  \item Die Gleichheitsrelation $=$ auf jeder Menge.
  \item Auf $\mathbb{Z}$: $a\equiv_n b :\Leftrightarrow n\mid(a-b)$ (Restklasse modulo $n$).
  \item Relation „sitzen in derselben Sitzreihe“ in einem Kinosaal.
\end{itemize}

\subsection{Klein. und grösst. Äquivalenzrelation}
Auf jeder Menge \(A\) existiert:
\begin{itemize}[noitemsep]
  \item die \emph{kleinste} Äquivalenzrelation: die Gleichheitsrelation \(\{(a,a)\mid a\in A\}\).
  \item die \emph{grösste} Äquivalenzrelation: das ganze \(A\times A\) (alles ist äquivalent).
\end{itemize}

\subsection{Äquivalenzklassen und Faktormenge}
Für $a\in A$ ist die \emph{Äquivalenzklasse}
\[
[a]_\sim := \{x\in A \mid x\sim a\}.
\]
Die Menge aller Äquivalenzklassen heisst \emph{Faktormenge} $A\diagup{\sim} := \{[a]_\sim \mid a\in A\}$.
Jedes Element einer Äquivalenzklasse ist ein \emph{Repräsentant} dieser Klasse.

\subsection{Wichtige Eigenschaften}
Für eine Äquivalenzrelation $\sim$ und $a,b\in A$ sind äquivalent:
\begin{enumerate}
  \item $a\sim b$,
  \item $[a]_\sim = [b]_\sim$,
  \item $[a]_\sim \cap [b]_\sim \neq \varnothing$,
  \item $a\in [b]_\sim$,
  \item $b\in [a]_\sim$.
\end{enumerate}
Daraus folgt: Zwei Äquivalenzklassen sind entweder gleich oder disjunkt.

\subsection{Beispiele in $\mathbb{R}^2$}
\begin{itemize}
  \item $(a,b)\approx(c,d) := a=c$\\ Äquivalenzklassen = vertikale Geraden.
  \item $(a,b)\simeq(c,d) := \sqrt{a^2+b^2}=\sqrt{c^2+d^2}$\\Äquivalenzklassen = Kreise um den Ursprung.
  \item Auf $\mathbb{R}^2\setminus\{(0,0)\}$: $(a,b)\sim(c,d) := \exists r\in\mathbb{R}(ra,rb)=(c,d)$\\Äquivalenzklassen = Geraden durch den Ursprung.
\end{itemize}



\subsection{Partitionen}
Eine \emph{Partition} einer Menge \(A\) ist eine Menge \(\{A_i\}_{i\in I}\) paarweise disjunkter, nichtleerer Teilmengen mit
\[
\bigcup_{i\in I} A_i = A.
\]
Die \(A_i\) nennt man auch \emph{Blöcke} der Partition.

\subsubsection{Beispiele}
\begin{itemize}
  \item Gerade und ungerade natürliche Zahlen: \(A_0=\{2n\},\; A_1=\{2n+1\}\).
  \item Einzelmengen \(\{n\}\) liefern eine feine Partition.
  \item Es gibt Partitionen von \(\mathbb{N}\) in unendlich viele unendliche Blöcke.
\end{itemize}

\subsubsection{Induzierte Partition}
Ist \(\sim\) eine Äquivalenzrelation auf \(A\), so sind die Äquivalenzklassen \([a]_\sim\) die Blöcke der Partition \(A\diagup{\sim}\).
Insbesondere sind die Klassen nichtleer und paarweise disjunkt.

\subsubsection{Induzierte Äquivalenzrelation}
Ist \(P=\{A_i\}_{i\in I}\) eine Partition von \(A\), definiert
\[
a\sim b \quad:\Leftrightarrow\quad \exists i\in I\; (a\in A_i \wedge b\in A_i)
\]
eine Äquivalenzrelation auf \(A\) mit Quotientenmenge \(A\diagup{\sim} = P\).

\subsubsection{Äquivalenzrelationen und Funktionen}
Eine Relation \(\sim\) auf \(A\) ist genau dann eine Äquivalenzrelation, wenn es eine Menge \(B\) und eine Abbildung \(f\colon A\to B\) gibt mit
\[
x\sim y \quad\Leftrightarrow\quad f(x)=f(y).
\]
(Äquivalenzklassen sind dann die Urbilder einzelner Werte von \(f\).)



\vfill

\rule{0.3\linewidth}{0.25pt}
\scriptsize

Copyright \copyright\ 2025 Justin Iven Müller

\href{https://github.com/JustinIven/zhaw-cheatsheets}{github.com/JustinIven/zhaw-cheatsheets}


\end{multicols*}
\end{document}
