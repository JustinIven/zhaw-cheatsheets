\documentclass[10pt,landscape]{article}
\usepackage{multicol}
\usepackage{calc}
\usepackage{ifthen}
\usepackage[landscape, a4paper, top=3mm, left=3mm, right=3mm, bottom=3mm]{geometry}
\usepackage[
     pdftex,
     bookmarksopen=true,
     pdfauthor={Justin Iven Müller},
     pdftitle={Diskrete Mathematik Cheatsheet},
     colorlinks=true,
     linkcolor=blue,
     urlcolor=blue
]{hyperref}
\usepackage{amsmath, amssymb}
\usepackage{enumitem}
\usepackage{tikz}
\usetikzlibrary{shapes,positioning,arrows,fit,calc,graphs,graphs.standard,intersections}
\usepackage{pgfplots}
\usepgfplotslibrary{fillbetween}
\usepackage[ngerman]{babel}
\usepackage{faktor}
\usepackage{stmaryrd}


\begin{document}
% Turn off header and footer
\pagestyle{empty}
 

% Redefine section commands to use less space
\makeatletter
\renewcommand{\section}{\@startsection{section}{1}{0mm}%
                                {-1ex plus -.5ex minus -.2ex}%
                                {0.5ex plus .2ex}%x
                                {\color{blue}\normalfont\large\bfseries}}
\renewcommand{\subsection}{\@startsection{subsection}{2}{0mm}%
                                {-1explus -.5ex minus -.2ex}%
                                {0.5ex plus .2ex}%
                                {\normalfont\normalsize\bfseries}}
\renewcommand{\subsubsection}{\@startsection{subsubsection}{3}{0mm}%
                                {-1ex plus -.5ex minus -.2ex}%
                                {1ex plus .2ex}%
                                {\normalfont\small\bfseries}}
\makeatother

% Remove indentation in itemize/enumerate and set left margin to zero
\setlist[itemize]{leftmargin=*}
\setlist[enumerate]{leftmargin=*}

% Don't print section numbers
\setcounter{secnumdepth}{0}

% Set font size and alignment for the whole document
\raggedright
\footnotesize

% Set column separation
% \setlength{\columnseprule}{1pt}
% \def\columnseprulecolor{\color{blue}}

\begin{multicols*}{4}

\begin{center}
     \Large{\textbf{Diskrete Mathematik}} \\
\end{center}

\section{Zahlenmengen}
\begin{tabular}{@{}ll@{}}
\(\mathbb{N}\) & natürliche Zahlen \\
\(\mathbb{N}_0\) & natürliche Zahlen mit 0 \\
\(\mathbb{Z}\) & ganze Zahlen \\
\(\mathbb{Q}\) & rationale Zahlen \\
\(\mathbb{R}\) & reelle Zahlen \\
\(\mathbb{C}\) & komplexe Zahlen \\
\end{tabular}
\section{Aussagenlogik}
\begin{tabular}{@{} p{1.2cm}|p{4.5cm}| @{}}
Aussage & Ein Satz, der entweder wahr (w) oder falsch (f) ist. \\
Prädikat & Eine Aussage mit Variablen. {\it n}-stellige Prädikate. \\
\end{tabular}

\subsection{Grundidee}
Aus gegebenen Prädikaten/Aussagen lassen sich durch Junktoren neue Aussagen bilden. (z.\,B. Kombinationen mit \(\land,\lor,\lnot,\Rightarrow,\Leftrightarrow\)).

\subsection{Definitionen}
\begin{itemize}
  \item \textbf{Negation:} \(\lnot A\) ist genau dann wahr, wenn \(A\) falsch ist. (Doppelte Negation: \(A\Leftrightarrow\lnot\lnot A\).)
  \item \textbf{Konjunktion:} \(A\land B\) ist wahr genau dann, wenn \(A\) und \(B\) wahr sind. (assoziativ, kommutativ, idempotent)
  \item \textbf{Disjunktion:} \(A\lor B\) ist wahr, wenn mindestens eine der Aussagen wahr ist. (assoziativ, kommutativ, idempotent)
  \item \textbf{Implikation:} \(A\Rightarrow B\) ist äquivalent zu \(\lnot A\lor B\). (Kontraposition: \(A\Rightarrow B \Leftrightarrow \lnot B\Rightarrow\lnot A\).)
  \item \textbf{Äquivalenz:} \(A\Leftrightarrow B\) genau dann, wenn \(A\Rightarrow B \land B\Rightarrow A\).
\end{itemize}

\subsection{Wichtige Regeln}
\begin{itemize}
  \item \textbf{De Morgan:}\\ 
    \(\lnot(A\land B)\Leftrightarrow \lnot A\lor\lnot B\) 
    \(\lnot(A\lor B)\Leftrightarrow\lnot A\land\lnot B\)
  \item \textbf{Distributivität:}
    \(A\land(B\lor C)\Leftrightarrow (A\land B)\lor(A\land C)\)
    \(A\lor(B\land C)\Leftrightarrow (A\lor B)\land(A\lor C)\)
  \item \textbf{Syntaktische Bindung:} \(\lnot\) bindet stärker als \(\land,\lor\); diese binden stärker als \(\Rightarrow,\Leftrightarrow\).
  \item \textbf{Modus Ponens:} Aus \(A \land (A\Rightarrow B\)) folgt \(B\).
  \item \textbf{Transitivität:} Aus \((A\Rightarrow B) \land (B\Rightarrow C)\) folgt \(A\Rightarrow C\).
\end{itemize}

\subsection{Hinweis zur Redundanz}
Jeder Ausdruck mit den Junktoren $\lnot,\land,\lor,\Rightarrow$ lässt sich ausschliesslich mit \(\lnot\) und \(\lor\) darstellen. z.B.
$$A\land B \Leftrightarrow \lnot(\lnot A\lor\lnot B)$$
$$A\lor B \Leftrightarrow \lnot(\lnot A \land \lnot B)$$

\section{Quantoren}
Quantoren dienen zur Formalisierung von Aussagen wie:
\begin{itemize}
  \item $\forall x\,A(x)$: \emph{Für alle $x$ gilt $A(x)$}
  \item $\exists x\,A(x)$: \emph{Es existiert ein $x$ mit $A(x)$}
\end{itemize}

Mehrere gleichartige Quantoren:
$$\forall x,y\;A(x,y) \quad\text{statt}\quad \forall x\,\forall y\;A(x,y)$$


\subsection{Eingeschränkte Quantoren}
$$\forall x \in M\,A(x): \text{Für alle }x\in M \text{ gilt }A(x)$$
$$\exists x \in M\,A(x): \text{Es gibt }x\in M \text{ mit }A(x)$$

Auch möglich mit Relationen:
$$\forall x < y\,A(x) \quad \text{oder} \quad \exists x \le y\,A(x)$$

\subsection{Als Junktoren}
Für endliche Mengen $M = \{x_1, \dots, x_n\}$ gilt:
$$\forall x \in M\,A(x) \Leftrightarrow A(x_1)\land \dots \land A(x_n)$$
$$\exists x \in M\,A(x) \Leftrightarrow A(x_1)\lor \dots \lor A(x_n)$$

\subsection{Als Makros}
$$\exists x \in M\,A(x) \Leftrightarrow \exists x\,(x \in M \land A(x))$$
$$\forall x \in M\,A(x) \Leftrightarrow \forall x\,(x \in M \Rightarrow A(x))$$

\subsection{Zusammenhang mit Junktoren}
$$\neg \forall x\,A(x) \Leftrightarrow \exists x\,\neg A(x)
\quad\text{und}\quad
\neg \exists x\,A(x) \Leftrightarrow \forall x\,\neg A(x)$$
$$\forall x\,(A(x)\land B(x)) \Leftrightarrow (\forall x\,A(x)) \land (\forall x\,B(x))$$
$$\exists x\,(A(x)\lor B(x)) \Leftrightarrow (\exists x\,A(x)) \lor (\exists x\,B(x))$$

\subsection{Leere Quantoren}
Wenn $x$ in $B$ nicht vorkommt:
$$\forall x\,B \Leftrightarrow B, \quad \exists x\,B \Leftrightarrow B$$

\section{Mengen}
\begin{itemize}
  \item \textbf{Menge / Element:} Eine Menge fasst mathematische Objekte (Elemente) zu einem Ganzen zusammen. Für Menge \(X\) und Element \(y\) gilt \(y\in X\) bzw.\ \(y\notin X\).
  \item \textbf{Aufzählende Schreibweise:} \(\{x_1,\dots,x_n\}\) bezeichnet die Menge, die genau die genannten Elemente enthält. Die leere Menge heisst \(\varnothing\).
  \item \textbf{Extensionalitätsprinzip:} Zwei Mengen sind genau dann gleich, wenn sie dieselben Elemente haben:
  \[ A=B \iff \forall x\,(x\in A \Leftrightarrow x\in B). \]
  \item \textbf{Teilmenge:} \(A\subseteq B\) genau dann, wenn \(\forall x\,(x\in A \Rightarrow x\in B)\). Ist \(A\subseteq B\) und \(A\neq B\), so ist \(A\) eine \emph{echte} Teilmenge, geschrieben \(A\subset B\).
  \item \textbf{Folgerungen:} Mengen sind ungeordnet; Mehrfachaufzählung desselben Elements ändert die Menge nicht. Für jede Menge \(A\) gilt \(\varnothing\subseteq A\).
\end{itemize}

\subsection{Eindeutigkeit der leeren Menge}
Seien \(e_1,e_2\) leere Mengen. Dann ist für alle \(x\) die Aussage \(x\in e_1\) falsch, also ist die Implikation \(x\in e_1\Rightarrow x\in e_2\) wahr; somit \(e_1\subseteq e_2\). Analog \(e_2\subseteq e_1\). Nach Extensionalität folgt \(e_1=e_2\).


\subsection{Aussonderungsprinzip}
Ist \(A\) eine Menge und \(E(x)\) eine Eigenschaft, dann gilt:
\[
\{x \in A \mid E(x)\} = \text{Menge aller } x \in A \text{ mit } E(x).
\]
\[
a \in \{x \in A \mid E(x)\} \iff a \in A \land E(a)
\]

\textbf{Beispiele:}
\begin{itemize}
  \item Gerade Zahlen: \(\{x \in \mathbb{N} \mid \exists y \in \mathbb{N} (x = 2y)\}\)
  \item Primzahlen: \(\{x \in \mathbb{N} \mid \exists y \in \mathbb{N} (y > 1) \land \forall a,b \in \mathbb{N} (ab = x \Rightarrow x=a \lor x=b)\}\)
\end{itemize}

\subsection{Ersetzungsprinzip}
Ist \(A\) eine Menge und \(t(x)\) ein Ausdruck, so gilt:
\[ \{t(x) \mid x \in A\} = \text{Menge aller Werte von } t(x) \text{ mit } x \in A. \]
\[ a \in \{t(x) \mid x \in A\} \iff \exists x \in A (a = t(x)) \]

\textbf{Beispiele:}
\begin{itemize}
  \item Quadratzahlen: \(\{x^2 \mid x \in \mathbb{N}\}\)
  \item Ungerade Zahlen: \(\{2x + 1 \mid x \in \mathbb{N}\}\)
  \item Rationale Zahlen: \(\left\{\frac{a}{b} \mid a,b \in \mathbb{Z},\, b \neq 0\right\}\)
  \item Anfangsabschnitte von \(\mathbb{N}\): \(\{\{x \in \mathbb{N} \mid x < y\} \mid y \in \mathbb{N}\}\)
\end{itemize}




\subsection{Vereinigung}
Die Vereinigung von zwei Mengen beinhaltet genau die Elemente, die in mindestens einer der beiden Mengen enthalten sind:
\[
A\cup B:=\{x\mid x\in A\vee x\in B\}.
\]

\subsection{Schnitt}
Die Schnittmenge von zwei Mengen beinhaltet genau die Elemente, die in beiden Mengen enthalten sind:
\[
A\cap B:=\{x\mid x\in A\land x\in B\}.
\]

\subsection{Allgemeine Vereinigung / Schnitt}
Sei \(I\) eine beliebige Indexmenge (z.\,B. \(I = \{1,2,\dots,n\}\) oder \(I = \mathbb{N}\)).  
Für jedes \(i \in I\) sei \(A_i\) eine Menge.

\subsubsection{Allgemeine Vereinigung}
\(x\) gehört zur Vereinigung genau dann, wenn es in \emph{mindestens einer} der Mengen \(A_i\) enthalten ist.
\[
\bigcup_{i \in I} A_i := \{\, x \mid \exists i \in I (x \in A_i)\}.
\]


\subsubsection{Allgemeiner Schnitt}
\(x\) gehört zum Schnitt genau dann, wenn es in \emph{allen} Mengen \(A_i\) enthalten ist.
\[
\bigcap_{i \in I} A_i := \{\, x \mid \forall i \in I (x \in A_i)\}.
\]

\subsection{Differenz}
Die Differenz von zwei Mengen beinhaltet genau die Elemente, die in der ersten Menge, aber nicht in der zweiten Menge enthalten sind:
\[
A\setminus B:=\{x\in A\mid x\notin B\}.
\]

\subsection{Disjunkte Mengen}
Zwei Mengen \(A\) und \(B\) heissen disjunkt, wenn sie keine gemeinsamen Elemente besitzen.
\[
A \cap B = \varnothing.
\]

\subsection{Paarweise disjunkt}
Eine Familie von Mengen \((A_i)_{i \in I}\) heisst paarweise disjunkt, wenn keine zwei verschiedenen Mengen ein gemeinsames Element haben. Es gilt:
\[
\forall i,j \in I \; (i \neq j \Rightarrow A_i \cap A_j = \varnothing).
\]

\subsection{Wichtige Eigenschaften}
Für beliebige Mengen \(A,B,C\) gelten:
\begin{itemize}
  \item Idempotenz: \(A\cup A=A,\; A\cap A=A\).
  \item Kommutativität: \(A\cup B=B\cup A,\; A\cap B=B\cap A\).
  \item Assoziativität: \(A\cup(B\cup C)=(A\cup B)\cup C\) und analog für \(\cap\).
  \item Teilmengen: \(A\subseteq A\cup B\) und \(A\cap B\subseteq A\).
  \item Distributivität:
  \[
  A\cup(B\cap C)=(A\cup B)\cap(A\cup C),
  \]
  \[
  A\cap(B\cup C)=(A\cap B)\cup(A\cap C).
  \]
  \item De Morgansche Regeln:
  \[
  C\setminus(A\cap B)=(C\setminus A)\cup(C\setminus B),
  \]
  \[
  C\setminus(A\cup B)=(C\setminus A)\cap(C\setminus B).
  \]
\end{itemize}

\subsection{Venn-Diagramm}
% Set A and B
\begin{tikzpicture}[scale=.45]
    \begin{scope}
        \clip \firstcircle;
        \fill[filled] \secondcircle;
    \end{scope}
    \draw[outline] \firstcircle node {$A$};
    \draw[outline] \secondcircle node {$B$};
    \node[anchor=south] at (current bounding box.north) {$A \cap B$};
\end{tikzpicture}
%Set A or B but not (A and B) also known a A xor B
\begin{tikzpicture}[scale=.45]
    \draw[filled, even odd rule] \firstcircle node {$A$}
                                 \secondcircle node{$B$};
    \node[anchor=south] at (current bounding box.north) {$\overline{A \cap B}$};
\end{tikzpicture}
% Set A or B
\begin{tikzpicture}[scale=.45]
    \draw[filled] \firstcircle node {$A$}
                  \secondcircle node {$B$};
    \node[anchor=south] at (current bounding box.north) {$A \cup B$};
\end{tikzpicture}
% Set A but not B
\begin{tikzpicture}[scale=.45]
    \begin{scope}
        \clip \firstcircle;
        \draw[filled, even odd rule] \firstcircle node {$A$}
                                     \secondcircle;
    \end{scope}
    \draw[outline] \firstcircle
                   \secondcircle node {$B$};
    \node[anchor=south] at (current bounding box.north) {$A\setminus B$};
\end{tikzpicture}

\subsection{Potenzmenge}
Für eine Menge \(A\) bezeichnet die Potenzmenge \( \mathcal{P}(A) \) die Menge aller Teilmengen von \(A\):
\[
\mathcal{P}(A):=\{X \mid X\subseteq A\}
\]

\textbf{Beispiele:}
\begin{align*}
\mathcal{P}(\{1,2\}) &= \{\varnothing,\{1\},\{2\},\{1,2\}\},\\
\mathcal{P}(\varnothing) &= \{\varnothing\},\\
\mathcal{P}(\{\{a\}\}) &= \{\varnothing,\{\{a\}\}\}.
\end{align*}

\textbf{Eigenschaften:}
\begin{itemize}
  \item \(A\in\mathcal{P}(A)\) und \(\varnothing\in\mathcal{P}(A)\).
  \item Aus \(A\subseteq B\) folgt \(\mathcal{P}(A)\subseteq\mathcal{P}(B)\).
  \item Für die leere Menge gilt \(\mathcal{P}(\varnothing)=\{\varnothing\}\neq\varnothing\).
  \item \(\mathcal{P}(A\cap B)=\mathcal{P}(A)\cap\mathcal{P}(B)\)
  \item \(\mathcal{P}(A\cup B)\supseteq \mathcal{P}(A)\cup\mathcal{P}(B)\)
  \item \(\mathcal{P}(A)\) einer endlichen Menge mit \(|A|=n\) hat \(|\mathcal{P}(A)|=2^n\) Elemente.
\end{itemize}

\subsection{Tupel}
Ein \(n\)-Tupel ist ein \emph{geordneter} Vektor
\[
  (a_1,\dots,a_n).
\]
Der \(i\)-te Eintrag eines Tupels \(a=(a_1,\dots,a_n)\) wird mit \(a[i]\) bezeichnet.
Zwei Tupel sind genau dann gleich, wenn sie dieselbe Länge haben und alle entsprechenden Einträge übereinstimmen:

\begin{multline*}
  (a_1,\dots,a_n)=(b_1,\dots,b_k) \iff \\
  n=k\ \land\ a_1=b_1\ \land \dots \land a_n=b_k
\end{multline*}

\subsubsection{Kartesisches Produkt}
Das kartesische Produkt \(A_1\times\cdots\times A_n\) ist die Menge aller \(n\)-Tupel, deren Einträge aus den Mengen \(A_1,\dots,A_n\) stammen.
\[
  A_1\times\cdots\times A_n:=\{(a_1,\dots,a_n)\mid a_i\in A_i\ \text{für }1\le i\le n\}.
\]

\textbf{Besonderheiten:}
\begin{itemize}
  \item Für das $n$-fache Produkt von \(A\) mit sich selbst gilt \(A^n:=A\times\cdots\times A\) (n-mal).
  \item Für ein kartesisches Produkt von der Form \(A_1\times\cdots\times A_n\) wird auch die Kurzschreibweise \(\prod_{i=1}^n A_i\) verwendet.
\end{itemize}

\textbf{Beispiele:}
\begin{align*}
  \{1\} \times \{a, b\} &= \{(1, a), (1, b)\} \\
  \mathbb{N}^2 &= \{(x, y) \mid x \in \mathbb{N} \land y \in \mathbb{N}\}
\end{align*}

\subsubsection{Projektionen}
Für eine Menge \(A\) von \(n\)-Tupeln und ist \(k \le n\) eine natürliche Zahl, definiert man die \(k\)-te Projektion:
\[
  \operatorname{pr}_k(A) := \{ x[k] \mid x \in A \}.
\]

\textbf{Insbesondere gilt:}
\[
  \operatorname{pr}_k(A_1 \times \cdots \times A_n) = A_k.
\]
\textbf{Beispiele:}
\begin{align*}
  \operatorname{pr}_1(\{1,2\} \times \{a,b\}) &= \{1, 2\} \\
  \operatorname{pr}_2(\{1,2\} \times \{a,b\}) &= \{a, b\}
\end{align*}
\section{Relationen}
Eine \emph{Relation} von \(A\) nach \(B\) ist ein Tripel
\[
R=(G,A,B)
\]
wobei \(A\) die Quellmenge, \(B\) die Zielmenge und \(G\subseteq A\times B\) der \emph{Graph} von \(R\) ist. Ist \(A=B\), so heisst \(R\) \emph{homogen} auf \(A\).

\subsection{Notation}
Sei \(R=(G,A,B)\) eine Relation von \(A\) nach \(B\).
\begin{itemize}
  \item Ist \(G\) der Graph von \(R\), so schreibt man \(G_R\)
  \item Ist \((x,y)\in G\), dann schreibt man \(xRy\) (\(x\) steht in Relation zu \(y\) bezüglich \(R\)).
  \item Sind \(A\) und \(B\) Teilmengen von \(\mathbb{R}\), so kann man \(R\) auch als Menge von Punkten in der Ebene darstellen: \(\{(x,y)\mid xRy\}\).\\
    % 1. x^2 = y^2
    \begin{tikzpicture}[scale=.45]
      \begin{axis}[axis lines=middle, xmin=-2, xmax=2, ymin=-2, ymax=2, width=5cm, height=5cm]
        \addplot[domain=-3:3, samples=100, blue!30] {x};
        \addplot[domain=-3:3, samples=100, blue!30] {-x};
      \end{axis}
      \node[anchor=south] at (current bounding box.north) {\tiny{$xRy: \Leftrightarrow x^2 = y^2$}};
    \end{tikzpicture}
    % 2. x^2 + y^2 = 1
    \begin{tikzpicture}[scale=.45]
      \begin{axis}[axis lines=middle, xmin=-2, xmax=2, ymin=-2, ymax=2, width=5cm, height=5cm]
        \addplot[domain=0:360, samples=200, blue!30] ({cos(x)}, {sin(x)});
      \end{axis}
      \node[anchor=south] at (current bounding box.north) {\tiny{$xRy: \Leftrightarrow x^2 + y^2 = 1$}};
    \end{tikzpicture}
  \item Als gerichteter Graph: Elemente von \(A\) und \(B\) als Knoten; für jedes \((x,y)\in G\) ein Pfeil \(x\to y\).\\
    \begin{tikzpicture}[main/.style = {draw, circle}] 
      \node[main] (1) {$1$}; 
      \node[main] (2) [right=0.5cm of 1] {$2$}; 
      \node[main] (3) [below=0.5cm of 1] {$3$}; 
      \node[main] (4) [right=0.5cm of 3] {$4$};
      \draw[->] (1) to (2);
      \draw[->] (1) to (3);
      \draw[->] (1) to (4);
      \draw[->] (2) to (4);
      \draw[->] (1) to [out=90,in=140,looseness=4] (1);
      \draw[->] (2) to [out=90,in=140,looseness=4] (2);
      \draw[->] (3) to [out=180,in=230,looseness=4] (3);
      \draw[->] (4) to [out=180,in=230,looseness=4] (4);
      \node[anchor=south] at (current bounding box.north) {\tiny{$xRy: \Leftrightarrow x \text{ teilt } y$}};
    \end{tikzpicture}
    \begin{tikzpicture}[main/.style = {draw, circle}] 
      \node[main] (1) {$1$}; 
      \node[main] (2) [right=0.5cm of 1] {$2$}; 
      \node[main] (3) [below=0.5cm of 1] {$3$}; 
      \node[main] (4) [right=0.5cm of 3] {$4$};
      \draw[<->] (1) to (3);
      \draw[<->] (2) to (4);
      \draw[->] (1) to [out=90,in=140,looseness=4] (1);
      \draw[->] (2) to [out=90,in=140,looseness=4] (2);
      \draw[->] (3) to [out=180,in=230,looseness=4] (3);
      \draw[->] (4) to [out=180,in=230,looseness=4] (4);
      \node[anchor=south] at (current bounding box.north) {\tiny{$xRy: \Leftrightarrow x+y \text{ ist gerade}$}};
    \end{tikzpicture} 
\end{itemize}


\subsection{Domäne und Bild}
Die Domäne und das Bild einer Relation geben an, welche Elemente der Quell- bzw. Zielmenge tatsächlich in der Relation vorkommen.
\begin{align*}
  \operatorname{dom}(R)&:=\operatorname{pr}_1(G_R)=\{a\in A\mid \exists b\in B(aRb)\}\\
  \operatorname{im}(R)&:=\operatorname{pr}_2(G_R)=\{b\in B\mid \exists a\in A(aRb)\}
\end{align*}
Im gerichteten Graphen entsprechen die Elemente der Domäne den Knoten mit ausgehenden Kanten, die des Bildes den Knoten mit eingehenden Kanten.


\subsection{Klassifizierungen}
Sei \(R\subseteq A\times A\) eine (homogene) Relation auf \(A\).

\subsubsection{Reflexivität}
Eine Relation \(R\) heisst \emph{reflexiv}, wenn jedes Element in Relation zu sich selbst steht:
\[\forall x\in A(xRx)\]
\begin{itemize}
    \item \(\{(a,a)\mid a\in A\}\subseteq R\).
    \item Im gerichteten Graphen hat jeder Knoten eine Kante zu sich selbst. Für jeden Wert \(x\in A\) gilt:
      \begin{tikzpicture}[main/.style = {draw, circle}] 
        \node[main] (x) {$x$};
        \draw[->] (x) to [out=90,in=140,looseness=4] (x);
      \end{tikzpicture}
    \item In der Koordinatendarstellung enthält \(R\) die Winkelhalbierende \(y=x\).
\end{itemize}

\subsubsection{Symmetrie}
Eine Relation \(R\) heisst \emph{symmetrisch}, wenn für alle \(x,y\in A\) gilt:
\[
  \forall x,y\;(xRy\Rightarrow yRx).
\]
\begin{itemize}
  \item Zu jedem Pfeil im gerichteten Graph existiert der umgekehrte Pfeil. Für alle \(x,y\in A\) gilt:\\
    \begin{tikzpicture}[main/.style = {draw, circle}] 
      \node[main] (x) {$x$};
      \node[main] (y) [right=0.5cm of x] {$y$};
      \draw[<->] (x) to (y);
    \end{tikzpicture}
  \item Symmetrie spiegelt die Koordinatendarstellung an der Geraden \(y=x\).
\end{itemize}

\subsubsection{Antisymmetrie}
Eine Relation \(R\) heisst \emph{antisymmetrisch}, wenn für
alle \(x,y\in A\) gilt:
\[
    \forall x,y\;(xRy\land yRx\Rightarrow x=y).
\]
\begin{itemize}
  \item Es gibt keine zwei verschiedenen Knoten, die wechselseitig verbunden sind.
    Für alle \(x,y\in A, x\neq y\) gilt:\\
    \begin{tikzpicture}[main/.style = {draw, circle}] 
      \node[main] (x) {$x$};
      \node[main] (y) [right=0.5cm of x] {$y$};
      \draw[->] (x) to (y);
    \end{tikzpicture}
\end{itemize}

\subsubsection{Transitivität}
Eine Relation \(R\) heisst \emph{transitiv}, wenn für jeden endlichen Pfad ein direkter Pfeil existiert. Für alle \(x,y,z\in A\) gilt:
\[
  \forall x,y,z\;(xRy\land yRz\Rightarrow xRz).
\]
\begin{itemize}
    \item Im gerichteten Graphen: Aus \(x\to y\) und \(y\to z\) folgt \(x\to z\). Für alle \(x,y,z\in A\) gilt:\\
      \begin{tikzpicture}[main/.style = {draw, circle}] 
        \node[main] (x) {$x$};
        \node[main] (y) [right=0.5cm of x] {$y$};
        \node[main] (z) [right=0.5cm of y] {$z$};
        \draw[->] (x) to (y);
        \draw[->] (y) to (z);
        \draw[->, dashed] (x) to [bend left] (z);
      \end{tikzpicture}
\end{itemize}

\subsubsection{Totalität und Eindeutigkeit}
Sei \(R\subseteq A\times B\) eine Relation von \(A\) nach \(B\) mit
\begin{itemize}
  \item \textbf{Linksvollständig / linkstotal:} \(\mathrm{dom}(R)=A\) (jedes Element in \(A\) hat min. eine \emph{ausgehende} Kante).
  \item \textbf{Rechtsvollständig / rechtstotal:} \(\mathrm{im}(R)=B\) (jedes Element in \(B\) hat min. eine \emph{eingehende} Kante).
  \item \textbf{Linkseindeutig:} \(\forall x_1,x_2,y\;(x_1Ry\land x_2Ry\Rightarrow x_1=x_2)\) (jedes Element in \(B\) hat max. eine \emph{eingehende} Kante).
  \item \textbf{Rechtseindeutig:} \(\forall x,y_1,y_2\;(xRy_1\land xRy_2\Rightarrow y_1=y_2)\) (jedes Element in \(A\) hat max. eine \emph{ausgehende} Kante).
\end{itemize}

\subsection{Inverse Relationen}
Für eine Relation $R = (G, A, B)$ ist die \emph{inverse Relation} definiert durch
\[
R^{-1} = (G', B, A), \quad G' := \{ (y,x) \mid (x,y) \in G \}.
\]
\textbf{Eigenschaften:}  
\begin{itemize}
    \item $(R^{-1})^{-1} = R$
    \item $R$ ist linksvollständig $\Leftrightarrow R^{-1}$ ist rechtsvollständig
    \item $R$ ist linkseindeutig $\Leftrightarrow R^{-1}$ ist rechtseindeutig
    \item Für jede symmetrische Relation $R$ gilt $R = R^{-1}$
\end{itemize}
\section{Funktionen}
Eine \emph{Funktion} \(f\) von der Menge \(A\) nach \(B\) ist eine Relation, die \emph{linksvollständig} und \emph{rechtseindeutig} ist. Man schreibt:
\[
f: A \to B,
\]
und für jedes \(x\in A\) existiert genau ein \(y\in B\) mit \(y=f(x)\).

\subsection{Schreibweise}
Oft werden Funktionen durch Angabe von Definitions- und Zielmenge sowie einer
Zuordnungsvorschrift beschrieben. Beispielsweise gilt:
\[
f = \bigl(\{(x, x^3) \mid x \in \mathbb{N}\}, \mathbb{N}, \mathbb{N}\bigr)
\]
bzw. äquivalent in der gebräuchlicheren Schreibweise:
\[
f : \mathbb{N} \to \mathbb{N}, \quad f(x) = x^3.
\]

\subsection{Injektive Funktionen}
Eine Funktion $f : A \to B$ ist \emph{injektiv}, falls die Relation \emph{linksvollständig}, \emph{rechtseindeutig} und zusätzlich \emph{linkseindeutig} ist:
\begin{align*}
  \forall x_1, x_2 &\in A (f(x_1) = f(x_2) \Rightarrow x_1 = x_2)\\
  \forall x_1, x_2 &\in A (x_1 \neq x_2 \Rightarrow f(x_1) \neq f(x_2))
\end{align*}
Jedes Element in \(A\) wird auf ein eigenes unterschiedliches Element in \(B\) abgebildet.

Notation: $f : A \hookrightarrow B$.

\subsection{Umkehrbarkeit}
Eine Funktion $f : A \to B$ ist genau dann \emph{umkehrbar}, wenn sie injektiv ist. Dann gilt:
\[
f^{-1} : \operatorname{im}(f) \to A.
\]
\[
(G^\prime_f, \operatorname{im}(f), A), \quad G^\prime_f = \{(y,x) | (x,y) \in G_f\}
\]

\subsection{Surjektivität}
Eine Funktion $f : A \to B$ ist \emph{surjektiv}, falls die Relation \emph{linksvollständig}, \emph{rechtseindeutig} und zusätzlich \emph{rechtsvollständig} ist:
\[
\operatorname{im}(f) = B
\]
Jedes Element in \(B\) wird von mindestens einem Element in \(A\) erreicht.\\
Notation: \(f:A \twoheadrightarrow B\)

\subsection{Bijektivität}
Eine Funktion $f : A \to B$ ist \emph{bijektiv}, wenn sie sowohl injektiv als auch surjektiv ist. Die Umkehrfunktion ist dann definiert durch:
\[
f^{-1} : B \to A.
\]
J
Notation: \(f : A \rightleftharpoons B\)


\subsection{Umkehrfunktion}
Für eine bijektive Funktion $f : A \rightleftharpoons B$ gilt:
\[
f^{-1} \circ f = \operatorname{id}_A, \qquad f \circ f^{-1} = \operatorname{id}_B.
\]

\subsection{Komposition}
Für $g : A \to B$ und $f : B \to C$ definiert man die \emph{Komposition}:
\[
(f \circ g)(x) = f(g(x)), \quad f \circ g : A \to C.
\]
Komposition ist \emph{assoziativ}:
\[
h \circ (g \circ f) = (h \circ g) \circ f.
\]

\subsection{Eigenschaften der Komposition}
Für Funktionen $f : A \to B$ und $g : B \to C$ gilt:
\begin{itemize}
  \item Sind $f$ und $g$ injektiv, dann ist $g \circ f$ injektiv.
  \item Sind $f$ und $g$ surjektiv, dann ist $g \circ f$ surjektiv.
  \item Sind $f$ und $g$ bijektiv, dann ist $g \circ f$ bijektiv.
\end{itemize}


\section{Äquivalenzrelationen}
Eine Relation $\sim$ auf einer Menge $A$ heisst \emph{Äquivalenzrelation}, falls sie für alle $x,y,z\in A$ die folgenden Eigenschaften erfüllt:
\begin{itemize}
  \item \textbf{reflexiv:} \quad $x\sim x$,
  \item \textbf{symmetrisch:} \quad $x\sim y \Rightarrow y\sim x$,
  \item \textbf{transitiv:} \quad $x\sim y \land y\sim z \Rightarrow x\sim z$.
\end{itemize}

\subsection{Beispiele}
\begin{itemize}
  \item Die Gleichheitsrelation $=$ auf jeder Menge.
  \item Auf $\mathbb{Z}$: $a\equiv_n b :\Leftrightarrow n\mid(a-b)$ (Restklasse modulo $n$).
  \item Relation „sitzen in derselben Sitzreihe“ in einem Kinosaal.
\end{itemize}

\subsection{Klein. und grösst. Äquivalenzrelation}
Auf jeder Menge \(A\) existiert:
\begin{itemize}[noitemsep]
  \item die \emph{kleinste} Äquivalenzrelation: die Gleichheitsrelation \(\{(a,a)\mid a\in A\}\).
  \item die \emph{grösste} Äquivalenzrelation: das ganze \(A\times A\) (alles ist äquivalent).
\end{itemize}

\subsection{Äquivalenzklassen und Faktormenge}
Für $a\in A$ ist die \emph{Äquivalenzklasse}
\[
[a]_\sim := \{x\in A \mid x\sim a\}.
\]
Die Menge aller Äquivalenzklassen heisst \emph{Faktormenge} $A\diagup{\sim} := \{[a]_\sim \mid a\in A\}$.
Jedes Element einer Äquivalenzklasse ist ein \emph{Repräsentant} dieser Klasse.

\subsection{Wichtige Eigenschaften}
Für eine Äquivalenzrelation $\sim$ und $a,b\in A$ sind äquivalent:
\begin{enumerate}
  \item $a\sim b$,
  \item $[a]_\sim = [b]_\sim$,
  \item $[a]_\sim \cap [b]_\sim \neq \varnothing$,
  \item $a\in [b]_\sim$,
  \item $b\in [a]_\sim$.
\end{enumerate}
Daraus folgt: Zwei Äquivalenzklassen sind entweder gleich oder disjunkt.

\subsection{Beispiele in $\mathbb{R}^2$}
\begin{itemize}
  \item $(a,b)\approx(c,d) := a=c$\\ Äquivalenzklassen = vertikale Geraden.
  \item $(a,b)\simeq(c,d) := \sqrt{a^2+b^2}=\sqrt{c^2+d^2}$\\Äquivalenzklassen = Kreise um den Ursprung.
  \item Auf $\mathbb{R}^2\setminus\{(0,0)\}$: $(a,b)\sim(c,d) := \exists r\in\mathbb{R}(ra,rb)=(c,d)$\\Äquivalenzklassen = Geraden durch den Ursprung.
\end{itemize}



\subsection{Partitionen}
Eine \emph{Partition} einer Menge \(A\) ist eine Menge \(\{A_i\}_{i\in I}\) paarweise disjunkter, nichtleerer Teilmengen mit
\[
\bigcup_{i\in I} A_i = A.
\]
Die \(A_i\) nennt man auch \emph{Blöcke} der Partition.

\subsubsection{Beispiele}
\begin{itemize}
  \item Gerade und ungerade natürliche Zahlen: \(A_0=\{2n\},\; A_1=\{2n+1\}\).
  \item Einzelmengen \(\{n\}\) liefern eine feine Partition.
  \item Es gibt Partitionen von \(\mathbb{N}\) in unendlich viele unendliche Blöcke.
\end{itemize}

\subsubsection{Induzierte Partition}
Ist \(\sim\) eine Äquivalenzrelation auf \(A\), so sind die Äquivalenzklassen \([a]_\sim\) die Blöcke der Partition \(A\diagup{\sim}\).
Insbesondere sind die Klassen nichtleer und paarweise disjunkt.

\subsubsection{Induzierte Äquivalenzrelation}
Ist \(P=\{A_i\}_{i\in I}\) eine Partition von \(A\), definiert
\[
a\sim b \quad:\Leftrightarrow\quad \exists i\in I\; (a\in A_i \wedge b\in A_i)
\]
eine Äquivalenzrelation auf \(A\) mit Quotientenmenge \(A\diagup{\sim} = P\).

\subsubsection{Äquivalenzrelationen und Funktionen}
Eine Relation \(\sim\) auf \(A\) ist genau dann eine Äquivalenzrelation, wenn es eine Menge \(B\) und eine Abbildung \(f\colon A\to B\) gibt mit
\[
x\sim y \quad\Leftrightarrow\quad f(x)=f(y).
\]
(Äquivalenzklassen sind dann die Urbilder einzelner Werte von \(f\).)


\input{content/posets.tex}
\section{Unendliche Mengen}
Zwei Mengen \(A\) und \(B\) haben dieselbe Mächtigkeit (Kardinalität) \(|A|=|B|\), genau dann, wenn eine bijektive Abbildung \(f:A\longleftrightarrow B\) existiert. Eine Menge heisst endlich, falls sie bijektiv zu \(\{1,\dots,n\}\) für ein \(n\in\mathbb{N}\) ist; andernfalls heisst sie unendlich.

\subsection{Wichtige Definitionen}
\begin{itemize}
  \item \(|A|=|B|\) : Existenz einer Bijektion \(f:A\to B\).
  \item \(|A|\le|B|\) : Es existiert eine injektive Abbildung \(g:A\hookrightarrow B\) (äquivalent: surjektive Abbildung \(B\twoheadrightarrow A\)).
  \item \(|A|=\infty\) : Abkürzung dafür, dass \(A\) nicht endlich ist.
  \item \(|\varnothing|\le|A|\) für alle Mengen \(A\).
\end{itemize}

\subsection{Elementare Eigenschaften}
\begin{itemize}
  \item Die Relation \(\sim\) mit \(A\sim B \Leftrightarrow |A|=|B|\) ist eine Äquivalenzrelation.
  \item Für endliche Mengen \(A\) und \(B\) mit \(|A|=n\) und \(|B|=m\) gilt \(|A|\le|B|\iff n\le m\).
  \item Eine Menge \(A\) ist genau dann unendlich, wenn \(|\mathbb{N}|\le|A|\).
  \item \(A\subseteq B \Rightarrow |A|\le|B|\). Umgekehrt: \(|A|\le|B|\) genau dann, wenn es \(A'\subseteq B\) mit \(|A'|=|A|\) gibt.
\end{itemize}

\subsection{Satz von Cantor--Bernstein}
Sind \(A\) und \(B\) nichtleer, dann gilt
\[
(|A|\le|B|\ \wedge\ |B|\le|A|)\ \Longleftrightarrow\ |A|=|B|.
\]
(Dieser Satz liefert aus beidseitigen Injektionen eine Bijektion.)

\subsection{Wichtige Folgerungen}
\begin{itemize}
  \item \textbf{Schubfachprinzip (Pigeonhole):} Aus \(|A|\le|B|\) und \(|A|\neq|B|\) folgt \(|B|\not\le|A|\).
  \item \textbf{Dedekind:} \(A\) ist unendlich \(\Leftrightarrow\) es existiert eine injektive, nicht surjektive Abbildung \(f:A\hookrightarrow A\). z.B. die Abbildung \(f:\mathbb{N}, n\mapsto n+1\).
  \item \textbf{Hilbert's Hotel (Anschaulichkeit):} Eine Menge \(A\) ist unendlich genau dann, wenn es eine echte Teilmenge \(B\subset A\) mit \(|B|=|A|\) gibt.
\end{itemize}

\section{Abzählbare Mengen}
Eine Menge \(A\) heisst \emph{abzählbar}, wenn \(A=\varnothing\) oder eine der folgenden (äquivalenten) Bedingungen erfüllt ist:
\begin{itemize}
  \item \(|A|\le | \mathbb{N} |\)
  \item Es existiert eine surjektive Funktion \(f:\mathbb{N}\twoheadrightarrow A\).
  \item Es existiert eine injektive Funktion \(f:A\hookrightarrow\mathbb{N}\).
\end{itemize}
\emph{Bemerkung:} Ist \(A\) abzählbar und unendlich, so gilt \(|A|=|\mathbb{N}|\).

\subsection{Beispiele}
\begin{itemize}
  \item Die leere Menge \(\varnothing\) ist abzählbar.
  \item Jede Teilmenge \(A\subseteq\mathbb{N}\) ist abzählbar (insbesondere \(\mathbb{N}\) selbst).
  \item \(\mathbb{Z}\) ist abzählbar.
  \item \(\mathbb{Q}\) ist abzählbar (als Schlussfolgerung aus Abzählbarkeit von \(\mathbb{Z}\times\mathbb{Z}\) und Quotientenbildung).
\end{itemize}

\subsection{Wichtige Aussagen}
\begin{itemize}
  \item Jede endliche Menge ist abzählbar. (Beweis: Aufzählung der Elemente liefert eine surjektive Abbildung \(\mathbb{N}\twoheadrightarrow A\).)
  \item Jede Teilmenge einer abzählbaren Menge ist abzählbar. (Bild- bzw. Einschränkungsargument.)
  \item Bild einer abzählbaren Menge unter einer surjektiven Abbildung ist abzählbar. (Komposition surjektiver Abbildungen.)
  \item \(\mathbb{N}\times\mathbb{N}\) ist abzählbar. Daraus folgt die Abzählbarkeit von \(\mathbb{Z}\times\mathbb{Z}\) und damit \(\mathbb{Q}\).\\
  \begin{center}
    \begin{tikzpicture}[scale=.8]
    \node (0-0) at (0, 0) {\tiny (0,0)};
    \node (1-0) at (1, 0) {\tiny (1,0)};
    \node (2-0) at (2, 0) {\tiny (2,0)};
    \node (3-0) at (3, 0) {\tiny (3,0)};
    \node (0-1) at (0, 1) {\tiny (0,1)};
    \node (1-1) at (1, 1) {\tiny (1,1)};
    \node (2-1) at (2, 1) {\tiny (2,1)};
    \node (2-0) at (2, 0) {\tiny (2,0)};
    \node (0-2) at (0, 2) {\tiny (0,2)};
    \draw[->] (0-0) -- (1-0);
    \draw[->] (1-0) -- (0-1);
    \draw[->] (0-1) -- (2-0);
    \draw[->] (2-0) -- (1-1);
    \draw[->] (1-1) -- (0-2);
    \draw[->] (0-2) -- (3-0);
    \draw[->] (3-0) -- (2-1);
    \draw[->, dashed] (2-1) -- (1,2);
  \end{tikzpicture}
  \end{center}
  \item Jede abzählbare Vereinigung abzählbarer Mengen \(\bigcup_{i\in\mathbb{N}} A_i\) ist abzählbar. (Beweisidee: doppelte Indizierung und Aufzählung aller Paare \((i,n)\).)
\end{itemize}

\section{Überabzählbare Mengen}
\begin{itemize}
  \item Es gibt verschiedene ``Grössen'' unendlicher Mengen; aus $|A|=\infty$ und $|B|=\infty$ folgt nicht notwendigerweise $|A|=|B|$.
  \item Es existiert eine unendliche \emph{Hierarchie} von Kardinalitäten: Mengen $A_0,A_1,A_2,\dots$ mit
    \[
      |A_0|<|A_1|<|A_2|<\dots
    \]
  \item Die Menge aller unendlichen Binärsequenzen $B=\{0,1\}^{\mathbb{N}}$ ist \emph{nicht abzählbar}.
\end{itemize}

\subsection{Sequenzen}
Eine \emph{Sequenz} in einer Menge $A$ ist eine Abbildung $s:\mathbb{N}\to A$. Die Menge aller Sequenzen in $A$ sei $A^{\mathbb{N}}$.\\
Entspricht \(s(0)=a_0, \, s(1)=a_1, \, s(2)=a_2, \, \dots\), so schreiben wir
\[  s = (a_0, a_1, a_2, \dots). \]

\subsection{Unendliche Binärsequenzen}
Eine \emph{Binärsequenz} ist eine Funktion $s:\mathbb{N}\to\{0,1\}$. Die Menge aller Binärsequenzen sei $B=\{0,1\}^{\mathbb{N}}$.

\subsection{Cantors Diagonalisierungsargument}
Für jede Abbildung $f:\mathbb{N}\to B$ konstruiere man $s\in B$ durch
\[
  s_n := 1 - f(n)_n
\]
\begin{align*}
    f(0) &= (\textcolor{red}{a^0_0}, a^0_1, a^0_2, \dots a^0_n, \dots) \\
    f(1) &= (a^1_0, \textcolor{red}{a^1_1}, a^1_2, \dots a^1_n, \dots) \\
    f(2) &= (a^2_0, a^2_1, \textcolor{red}{a^2_2}, \dots a^2_n, \dots) \\
    \vdots \\
    f(n) &= (a^n_0, a^n_1, a^n_2, \dots \textcolor{red}{a^n_n}, \dots) \\
    \vdots
\end{align*}
$s$ unterscheidet sich von jedem Bild $f(n)$ in der $\textcolor{red}{n}$-ten Stelle. Somit ist $s\notin\operatorname{im}(f)$ und es gibt keine surjektive Abbildung $\mathbb{N}\to B$. Daher ist $B$ nicht abzählbar.

\subsection{Folgerungen}
\begin{itemize}
  \item Das Intervall $[0,1)$ und damit $\mathbb{R}$ sind überabzählbar.
  \item Die Menge aller Funktionen $\mathbb{N}\to\mathbb{N}$ ist überabzählbar.
  \item Es existieren Funktionen $\mathbb{N}\to\mathbb{N}$, die nicht berechenbar sind.
\end{itemize}

\subsection{Potenzmenge und Cantors Theorem}
Für jede Menge $A$ gilt streng:
\[
  |A| < | \mathcal{P}(A) |.
\]
Begründung:
\begin{enumerate}
  \item Es existiert eine Injektion $A\hookrightarrow\mathcal{P}(A)$, $x\mapsto\{x\}$, also $|A|\le|\mathcal{P}(A)|$.
  \item Für jede Abbildung $f:A\to\mathcal{P}(A)$ betrachte die Menge
    \[
      \Delta_f := \{ a\in A \mid a\notin f(a)\}.
    \]
    $\Delta_f\in\mathcal{P}(A)$, aber $\Delta_f\notin\operatorname{im}(f)$ (diagonalisiertes Argument), also ist $f$ nicht surjektiv. Damit $|\mathcal{P}(A)|\nleq|A|$.
\end{enumerate}
\section{Peano-Axiome}
Die Peano-Axiome beschreiben die Grundstruktur der natürlichen Zahlen $\mathbb{N}$:
\begin{enumerate}[label=\textbf{Axiom \arabic*:}, leftmargin=*, itemsep=4pt]
  \item $0\in\mathbb{N}$.
  \item Zu jeder $k\in\mathbb{N}$ existiert genau ein Nachfolger $k+1\in\mathbb{N}$ (Nachfolgerfunktion $\eta : \mathbb{N} \to \mathbb{N}, \quad \eta(n)=n+1$ ).
  \item $0$ ist die einzige Zahl, die kein Nachfolger ist. \(\forall n\in\mathbb{N}\ (\forall k \in \mathbb{N}\ (n \neq k+1) \Leftrightarrow n=0)\)
  \item Die Nachfolgerfunktion \(\eta : \mathbb{N} \hookrightarrow \mathbb{N}\setminus\{0\}\) ist injektiv: $n+1=m+1\Rightarrow n=m$.
\end{enumerate}
\section{Induktion}

\subsection{Axiom der vollständigen Induktion}
Sei $A(n)$ eine Aussage für $n\in\mathbb{N}$. Gilt
\begin{itemize}
  \item \emph{Induktionsverankerung (I.V.)}: $A(0)$
  \item \emph{Induktionsschritt (I.S.)}: $\forall n\in\mathbb{N}\ (A(n)\Rightarrow A(n+1))$,
\end{itemize}
so folgt $\forall n\in(\mathbb{N}\ A(n))$.

\subsection{Varianten der vollständigen Induktion}
\begin{description}
  \item[Mengeninduktion] Für $A\subseteq\mathbb{N}$ gilt: Ist $0\in A$ und $\forall n\in\mathbb{N}(n\in A\Rightarrow n+1\in A)$, so ist $A=\mathbb{N}$.
  \item[Induktion mit Startwert] Für festen $z\in\mathbb{Z}$: Aus $A(z)$ und $\forall n\ge z\ (A(n)\Rightarrow A(n+1))$ folgt $\forall n\ge z\ A(n)$. Dies folgt durch Anwendung der normalen Induktion auf $B(n):=A(n+z)$.
  \item[Minimumsprinzip] Jede nichtleere Teilmenge $A\subseteq\mathbb{N}$ besitzt ein kleinstes Element. Dieses Prinzip ist äquivalent zum Induktionsprinzip.
  \item[Kleinster Verbrecher] Beweis per Widerspruch: Existiert ein kleinstes $n$ ohne Eigenschaft $A$, führt dies oft zu einem Widerspruch und damit zum Beweis von $\forall n\ A(n)$.
  \item[Starke Induktion] Gilt $\forall n\in\mathbb{N}\ \big(\forall m<n\ A(m)\Rightarrow A(n)\big)$, so folgt $\forall n\in\mathbb{N}\ A(n)$. Die starke Induktion ist logisch äquivalent zur gewöhnlichen Induktion.
  \item[Absteigende Ketten] Es existiert keine unendliche streng absteigende Folge $a_0>a_1>a_2>\dots$ in $\mathbb{N}$. Andernfalls würde die Menge $\{a_i\}$ kein Minimum besitzen, im Widerspruch zum Minimumsprinzip.
\end{description}

\subsection{Rekursion}
Rekursive Algorithmen folgen dem Prinzip: \emph{Problem} \(\to\) in kleinere ähnliche Teilprobleme zerlegen; Basisfälle direkt lösen; Teillösungen rekursiv berechnen und kombinieren.


\subsubsection{Rekursive Definitionen}
\begin{enumerate}[noitemsep]
  \item Man spezifiziert Basisfälle (einfachste Bestandteile).
  \item Man gibt Rekursionsschritte an, die aus bereits definierten (einfacheren) Fällen neue Fälle aufbauen.
  \item Die Gesamtheit dieser Regeln definiert das Objekt.
\end{enumerate}

\subsubsection{Primitive Rekursion (ohne Parameter)}
Seien \(M\) eine Menge, \(g: M\times\mathbb{N}\to M\) und \(c\in M\). Dann existiert eindeutig eine Funktion \(f:\mathbb{N}\to M\) mit
\[
\begin{aligned}
f(0)&=c,\\
f(n+1)&=g(f(n),n).
\end{aligned}
\]

\subsubsection{Primitive Rekursion (mit Parameter)}
Sind \(M,X\) Mengen, \(g: M\times\mathbb{N}\times X\to M\) und \(c:X\to M\), so gibt es eindeutig \(f:\mathbb{N}\times X\to M\) mit
\[
\begin{aligned}
f(0,x)&=c(x),\\
f(n+1,x)&=g(f(n,x),n,x)\quad(\forall x\in X).
\end{aligned}
\]

\subsubsection{Wichtige Beispiele (primitive Rekursion)}
\begin{itemize}[noitemsep]
  \item \textbf{Addition:}
  \[
  \begin{aligned}
  x+0&=x,\\
  x+(n+1)&=(x+n)+1.
  \end{aligned}
  \]
  \item \textbf{Multiplikation:}
  \[
  \begin{aligned}
  x\cdot 0 &= 0,\\
  x\cdot(n+1) &= (x\cdot n)+x.
  \end{aligned}
  \]
  \item \textbf{Exponentiation:}
  \[
  \begin{aligned}
  x^{0}&=1,\\
  x^{n+1}&=x\cdot x^n.
  \end{aligned}
  \]
  \item \textbf{Endliche Summen und Produkte} lassen sich ebenfalls rekursiv definieren (rekursiver Startwert und Schritt).
\end{itemize}

\subsection{Zusammenspiel von Rekursion und Induktion}
Rekursive Definitionen von Objekten (z.B. Summen, Produkte) erlauben Beweise über deren Eigenschaften (Kommutativität, Assoziativität, etc.) mittels Induktion.


\section{Strukturelle Induktion/Rekursion}
Induktive Mengen verallgemeinern die Struktur der natürlichen Zahlen. Statt eines speziellen Grundelements \(0\) und der Nachfolgerabbildung \(\eta(n)=n+1\) betrachtet man:
\begin{itemize}
  \item eine Menge von \emph{Grundelementen} \(A_0\subseteq M\),
  \item eine Menge von (n-stelligen) \emph{Regeln} \(R\), wobei jede Regel \(r\) eine Funktion \(r:M^n\to M\) ist.
\end{itemize}
Die induktive Menge \(N(A_0,R)\) ist die kleinste Teilmenge von \(M\), die \(A_0\) enthält und unter allen Regeln in \(R\) abgeschlossen ist.

\subsection{Abschlussregeln und Abgeschlossenheit}
Eine Menge \(A\subseteq M\) ist \emph{unter einer Regel} \(r:M^n\to M\) abgeschlossen, falls
\[
(x_1,\dots,x_n)\in A^n \;\Rightarrow\; r(x_1,\dots,x_n)\in A.
\]
Ist \(R\) eine Menge von Regeln, so ist \(A\) unter \(R\) abgeschlossen, wenn sie unter jeder Regel in \(R\) abgeschlossen ist.
Beispiele:
\begin{itemize}
  \item \(\mathbb{N}\) ist abgeschlossen unter \(\{+,\cdot\}\).
  \item \(\mathbb{Z}\) ist abgeschlossen unter \(\{+,-,\cdot\}\).
  \item Die Menge der geraden Zahlen ist abgeschlossen unter \(\{+,-,\cdot\}\).
\end{itemize}

\subsection{Existenz und Eindeutigkeit}
Für gegebene \(M\), \(A_0\subseteq M\) und Regelmenge \(R\) existiert eine eindeutige kleinste Menge
\begin{multline*}
  N(A_0,R):= \\
  \bigcap\{A\subseteq M \mid A_0\subseteq A \land A \text{ ist abg. unter } R \},  
\end{multline*}

die alle Grundelemente enthält und unter \(R\) abgeschlossen ist.

\subsection{Strukturelle Induktion}
Um eine Eigenschaft \(P(x)\) für alle \(x\in N(A_0,R)\) zu beweisen, reicht es zu zeigen:
\begin{enumerate}[label=(\alph*)]
    \item Für alle Grundelemente \(a\in A_0\) gilt \(P(a)\).
    \item Für jede Regel \(f\in R\) mit \(k\) Argumenten aus \(P(x_1),\dots,P(x_k)\) folgt \(P(f(x_1,\dots,x_k))\).
\end{enumerate}
Dann gilt \(P(x)\) für alle \(x\in N(A_0,R)\).

\subsection{Strukturelle Rekursion}
Strukturelle Rekursion definiert Funktionen auf \(N(A_0,R)\) durch Angabe:
\begin{itemize}
  \item Werte für alle Grundelemente \(a\in A_0\) (Basisfälle),
  \item Rekursionsgleichungen, die jedem Konstruktor \(f\in R\) eine Funktion \(g_f\) zuordnen, welche die Werte auf den Komponenten zu einem Wert für \(f(\dots)\) kombiniert.
\end{itemize}
Dies generalisiert primitive Rekursion auf \(\mathbb{N}\).

\subsection{Beispiele}
\subsubsection{Listen / Tupel \(A^*\)}
Induktive Definition: \(A^* := N(\{()\},\{ \text{cons}_a \mid a\in A \})\) mit \(\text{cons}_a(\ell)=(a,\ell)\).
\begin{itemize}
  \item Länge: 
     \begin{align*}
     \mathrm{len}(()) &:= 0, \\
     \mathrm{len}(\mathrm{cons}_a(\ell)) &:= 1+\mathrm{len}(\ell)
     \end{align*}
  \item Summe: 
    \begin{align*}
     \mathrm{sum}(()) &:= 0, \\
     \mathrm{sum}(\mathrm{cons}_a(\ell)) &:= a+\mathrm{sum}(\ell)
     \end{align*}
  \item Minimum:
    \begin{align*}
    \mathrm{min}(()) &:= \infty, \\
    \mathrm{min}(\mathrm{cons}_a(\ell)) &:= \min(a,\mathrm{min}(\ell)).
    \end{align*}
\end{itemize}

\subsubsection{Binärbäume \(\mathrm{tree}(A)\)}
Induktive Definition: \(\mathrm{tree}(A):=N(A,\{\mathrm{node}\})\) mit \(\mathrm{node}(x,y)=(x,y)\) und Blättern aus \(A\).
\begin{itemize}
  \item Tiefe: 
    \begin{align*}
    \mathrm{depth}(a) &:= 0, \\
    \mathrm{depth}(\mathrm{node}(x,y)) &:= 1+\max(\mathrm{depth}(x),\mathrm{depth}(y)).
    \end{align*}
  \item Blatt-Summe: 
    \begin{align*}
    \mathrm{sumLeaf}(a) &:= a, \\
    \mathrm{sumLeaf}(\mathrm{node}(x,y)) &:= \mathrm{sumLeaf}(x)+\mathrm{sumLeaf}(y).
    \end{align*}
\end{itemize}
\section{Formale Aussagenlogik}
Die Menge der aussagenlogischen Formeln wird induktiv definiert:
\begin{itemize}[nosep]
  \item Grundelemente: Variablen $x_1,x_2,\dots$
  \item Wenn $A,B$ Formeln sind, dann sind $(A\land B)$ und $(A\lor B)$ Formeln.
  \item Wenn $A$ eine Formel ist, dann ist $\lnot A$ eine Formel.
\end{itemize}

\subsection{Syntax}
Die Syntax der Aussagenlogik (Menge aller aussagenlogischen Formeln) wird durch die obige Induktion definiert. Sie ist gegeben durch \(N(\{x_1,x_2,\dots\}, \{\text{and}, \text{or}, \text{not}\})\) mit 
\begin{align*}
\text{and}(A,B) &:=(A \land B) \\
\quad \text{or}(A,B) &:=(A \lor B) \\
\quad \text{not}(A) &:=(\lnot A).
\end{align*}

\subsection{Syntaxbaum}
Jede Formel besitzt einen zugehörigen Syntaxbaum, der ihre rekursive Struktur darstellt.
\[(\lnot x_3 \lor x_2) \land \lnot (x_1 \lor x_0)\]
\begin{center}
\begin{tikzpicture}
    % Nodes
    \node (1) at (3,0) {$\land$};
    \node (2) at (2,-0.8) {$\lor$};
    \node (3) at (4,-0.8) {$\lnot$};
    \node (4) at (1.5,-1.6) {$\lnot$};
    \node (5) at (2.5,-1.6) {$x_2$};
    \node (6) at (4,-1.6) {$\lor$};
    \node (0) at (4.5,-2.4) {$x_0$};
    \node (7) at (3.5,-2.4) {$x_1$};
    \node (8) at (1.5,-2.4) {$x_3$};
    % Edges
    \draw (1) -- (2);
    \draw (1) -- (3);
    \draw (2) -- (4);
    \draw (2) -- (5);
    \draw (3) -- (6);
    \draw (4) -- (8);
    \draw (6) -- (7);
    \draw (6) -- (0);
\end{tikzpicture}
\end{center}

\subsection{Strukturelle Rekursion}
Strukturelle Rekursion ist wie primitive Rekursion, nur dass sie nicht über \(\mathbb{N}\), sondern über die jeweils echte Unterstruktur eines rekursiven Datentyps (z. B. Listen oder Bäume) definiert wird.

Funktionen auf Formeln werden begründet durch:
\begin{itemize}
  \item \emph{Basisfall}: für jede Variable \(x_i\) wird \(f(x_i)\) definiert.
  \item \emph{Rekursionsschritte}: für jede Verknüpfung (z.B. \(\land, \lor, \lnot\)) wird eine Funktion \(g_{\land}, g_{\lor}, g_{\lnot}\) definiert, die die Werte der Funktion auf den Teilformeln kombiniert.
\end{itemize}

\textbf{Beispiele:}
Die Menge aller Subfunktionen einer Formel ist durch strukturelle Rekursion definiert als:
\begin{align*}
\text{sufo}(x_i) &:= \{x_i\} \\
\text{sufo}(A \land B) &:= \{A \land B\} \cup \text{sufo}(A) \cup \text{sufo}(B) \\
\text{sufo}(A \lor B) &:= \{A \lor B\} \cup \text{sufo}(A) \cup \text{sufo}(B) \\
\text{sufo}(\lnot A) &:= \{\lnot A\} \cup \text{sufo}(A)
\end{align*}
Analog: Menge aller Variablen in einer Formel $\operatorname{vars}(\cdot)$.


\subsection{Semantik}
Jede Formel $A$ mit Variablen in $\{x_1,\dots,x_n\}$ wird als boolesche Funktion
$\llbracket  A \rrbracket_n:\{0,1\}^n\to\{0,1\}$ interpretiert:
\begin{align*}
  \llbracket x_i \rrbracket_n (b_1,\dots,b_n) &=
  \begin{cases} b_i & 1\le i\le n\\ 0 & \text{sonst}\end{cases} \\
  \llbracket  A\land B \rrbracket_n (b_1,\dots,b_n) &=\min(\llbracket A\rrbracket_n,\llbracket B\rrbracket_n), \\
  \llbracket  A\lor B \rrbracket_n (b_1,\dots,b_n) &=\max(\llbracket A\rrbracket_n,\llbracket B\rrbracket_n), \\
  \llbracket \lnot A \rrbracket_n (b_1,\dots,b_n) &=1-\llbracket A\rrbracket_n.
\end{align*}
        
\subsection{Semantische Eigenschaften}
Viele Eigenschaften von Formeln lassen sich über ihre semantische Interpretation definieren. Beispiele:
\begin{itemize}
  \item \emph{allgemeingültig}: $\llbracket A\rrbracket_n = (\vec{b} \mapsto 1)$ für alle $n \in \mathbb{N}$,
  \item \emph{unerfüllbar}: $\llbracket A\rrbracket_n = (\vec{b} \mapsto 0)$ für alle $n \in \mathbb{N}$,
  \item \emph{erfüllbar}: es existiert ein $n \in \mathbb{N}$ und ein Vektor $\vec{b} \in \{0,1\}^n$ mit $\llbracket A \rrbracket_n (\vec{b}) =1$,
  \item \emph{äquivalent}: $\llbracket A\rrbracket_n=\llbracket B\rrbracket_n$ für alle $n \in \mathbb{N}$.
\end{itemize}

\subsection{Wahrheitstabellen}
Die Semantik einer Formel kann durch Wahrheitstabellen dargestellt werden; die letzte Spalte gibt Werte von $\llbracket A\rrbracket_n$ an. Erfüllbarkeit, Allgemeingültigkeit und Äquivalenz lassen sich daraus ablesen. 

{
  \tiny
  \[
  \begin{array}{|c|c|c|c|c|}
  \hline
  \{0,1\}^2 & \llbracket x_1 \rrbracket_2 & \llbracket x_2 \rrbracket_2 & \llbracket \neg x_2 \rrbracket_2 & \llbracket (x_1 \wedge \neg x_2) \rrbracket_2 \\
  \hline
  (0,0) & 0 & 0 & 1 & 0 \\
  (1,0) & 1 & 0 & 1 & 1 \\
  (0,1) & 0 & 1 & 0 & 0 \\
  (1,1) & 1 & 1 & 0 & 0 \\
  \hline
  \end{array}
  \]
}

\subsection{Funktionale Vollständigkeit}
Für jede boolesche Funktion $f:\{0,1\}^n\to\{0,1\}$ existiert eine aussagenlogische Formel $A$ mit $\llbracket A\rrbracket_n=f$.
Diese kann z.B. durch die disjunktive Normalform (DNF) konstruiert werden.

\subsection{Normalformen}
Rekursive Definitionen für Mengen $K_n$ und $D_n$:
\begin{align*}
K_0=D_0&=\{x_i,\lnot x_i\mid i\in\mathbb{N}\},\\
K_{n+1}&=\{\bigwedge_{j} A_j \mid A_j\in D_n\},\\
D_{n+1}&=\{\bigvee_{j} A_j \mid A_j\in K_n\}.
\end{align*}
Es gilt für alle $n\in\mathbb{N}$:
\begin{align*}
  D_n &\subseteq K_{n+1}, \quad D_n \subseteq D_{n+1}, \\
  K_n &\subseteq K_{n+1}, \quad K_n \subseteq D_{n+1}.
\end{align*}

Formeln in $K_2$ sind in konjunktiver Normalform (KNF), solche in $D_2$ in disjunktiver Normalform (DNF).
Jede Formel \(\llbracket A \rrbracket_n\) besitzt äquivalente Darstellungen in KNF und DNF (Konstruktion über die Wertetabelle und De-Morgan-Transformationen).

\subsubsection{Konstruktion der DNF}
Die disjunktive Normalform (DNF) einer Formel \(A\) mit Variablen in \(\{x_1, \ldots, x_n\}\) wird konstruiert durch:
\begin{itemize}
  \item Bestimmen aller Belegungen \(\vec{b} \in \{0,1\}^n\), für \(\llbracket A \rrbracket_n(\vec{b}) = 1\).
  \item Für jede solche Belegung \(\vec{b} = (b_1, \ldots, b_n)\) wird ein Konjunktionsglied \(K_{\vec{b}}\) gebildet:
  \[K_{\vec{b}} = \bigwedge_{i=1}^{n} \begin{cases} x_i & \text{wenn } b_i = 1 \\ \lnot x_i & \text{wenn } b_i = 0 \end{cases}\]
  \item Die DNF von \(A\) ist dann die Disjunktion aller Konjunktionsglieder:
  \[\text{DNF}(A) = \bigvee_{\vec{b} \text{ mit } \llbracket A \rrbracket_n(\vec{b}) = 1} K_{\vec{b}}\]
\end{itemize}

\subsubsection{Konstruktion der KNF}
Die konjunktive Normalform (KNF) einer Formel \(A\) mit Variablen in \(\{x_1, \ldots, x_n\}\) wird konstruiert durch:
\begin{itemize}
  \item Bestimmen aller Belegungen \(\vec{b} \in \{0,1\}^n\), für \(\llbracket A \rrbracket_n(\vec{b}) = 0\).
  \item Für jede solche Belegung \(\vec{b} = (b_1, \ldots, b_n)\) wird ein Disjunktionsglied \(D_{\vec{b}}\) gebildet:
  \[D_{\vec{b}} = \bigvee_{i=1}^{n} \begin{cases} \lnot x_i & \text{wenn } b_i = 1 \\ x_i & \text{wenn } b_i = 0 \end{cases}\]
  \item Die KNF von \(A\) ist dann die Konjunktion aller Disjunktionsglieder:
  \[\text{KNF}(A) = \bigwedge_{\vec{b} \text{ mit } \llbracket A \rrbracket_n(\vec{b}) = 0} D_{\vec{b}}\]
\end{itemize}
\section{Elementare Zahlentheorie}
\subsection{Teilbarkeitsrelation}
Für ganze Zahlen \(x,y \in \mathbb{Z}\) heisst \(y\) ein Vielfaches von \(x\), wenn es ein \(t \in \mathbb{Z}\) mit \(y = tx\) gibt. In diesem Fall heisst \(x\) ein Teiler von \(y\) und man schreibt \(x | y\).
Die Menge der natürlichen Teiler einer Zahl \(x\) ist
\[
T(x) := \{ n \in \mathbb{N} \mid n | x \}.
\]

\subsection{Teilen mit Rest}
Für \(a,b \in \mathbb{Z}\) mit \(b \neq 0\) existieren eindeutig bestimmte ganze Zahlen \(m,r \in \mathbb{Z}\) mit
\[
a = mb + r, \qquad 0 \le r < |b|.
\]
Für natürliche Zahlen \(a,b \in \mathbb{N}\) mit \(b \neq 0\) gilt entsprechend mit \(m,r \in \mathbb{N}\)
\[
a = mb + r, \qquad r < b.
\]


\subsection{Ganzzahlige Division und Modulo}
Für \(a,b \in \mathbb{Z}\) mit \(b \neq 0\) werden die Funktionen
\begin{align*}
    \operatorname{div}: \mathbb{N} \times \mathbb{N} \setminus \{0\} &\to \mathbb{Z}, \\
    \operatorname{mod}: \mathbb{N} \times \mathbb{N} \setminus \{0\} &\to \mathbb{N}.
\end{align*}
durch
\[
a = \operatorname{div}(a,b)\cdot b + \operatorname{mod}(a,b), 
\qquad 0 \le \operatorname{mod}(a,b) < |b|
\]
definiert. Dabei entspricht \(\operatorname{div}\) der ganzzahligen Division und
\(\operatorname{mod}\) dem Rest.


\subsection{Grösster gemeinsamer Teiler}
Für ganze Zahlen \(a_1,\dots,a_n\) ist die Menge der gemeinsamen Teiler
\[
T(a_1,\dots,a_n) := T(a_1) \cap \dots \cap T(a_n).
\]
Der grösste gemeinsame Teiler ist definiert als
\[
\operatorname{ggT}(a_1,\dots,a_n) := \max T(a_1,\dots,a_n),
\]
sofern nicht alle Zahlen Null sind. Zwei Zahlen heissen \emph{teilerfremd}, wenn ihr
grösster gemeinsamer Teiler gleich \(1\) ist.


\subsection{Euklidischer Algorithmus}
Für beliebige ganze Zahlen \(a,b\) gilt
\[
\operatorname{T}(a,b) = \operatorname{T}(a,b-a).
\]
Für ganze Zahlen \(a, b\) die nicht beide Null sind, gilt
\[
\operatorname{ggT}(a,b) = \operatorname{ggT}(a,b-a).
\]

Daraus folgt allgemein die rekursive Definition des ggT. Für \((a,b) \in \mathbb{N}^2 \setminus \{(0,0)\}\) gilt:
\[
\operatorname{ggT}(a,b) = \begin{cases}
    \operatorname{max}(a,b) & \text{falls } a = 0 \lor b = 0 \\
    \operatorname{ggT}(\operatorname{mod}(a,b),b) & \text{falls } a \geq b \\
    \operatorname{ggT}(a,\operatorname{mod}(b,a)) & \text{sonst}
\end{cases}
\]
Durch Festlegung von \(\operatorname{ggT}(a,b) := \operatorname{ggT}(|a|,|b|)\) für
\(a,b \in \mathbb{Z}\) wird der euklidische Algorithmus auf alle ganzen Zahlen erweitert.


\subsection{Lemma von Bézout}
Für \(x,y \in \mathbb{Z}\), die nicht beide Null sind, existieren ganze Zahlen
\(a,b\) mit
\[
\operatorname{ggT}(x,y) = ax + by.
\]
Die Zahlen \(a\) und \(b\) heissen Bézout-Koeffizienten. Sie lassen sich mit dem
erweiterten euklidischen Algorithmus bestimmen.


\section{Primzahlen}
Eine Zahl \(p\in\mathbb{N}\) heisst Primzahl, wenn \(|T(p)|=2\); äquivalent dazu ist \(p>1\) und \(T(p)=\{1,p\}\). Die Menge aller Primzahlen wird mit \(\mathbb{P}\) bezeichnet.

\begin{itemize}
    \item \textbf{Existenz von Primfaktoren.} Zu jeder natürlichen Zahl \(n>1\) existiert eine Primzahl \(p\) mit \(p\mid n\). Daher lässt sich jede \(n>1\) als Produkt endlich vieler Primzahlen darstellen.
    \item \textbf{Unendlichkeit der Primzahlen.} Es existieren unendlich viele Primzahlen (klassisches Argument nach Euklid).
    \item \textbf{Euklidsches Lemma.} Für \(p\in\mathbb{N}\) sind äquivalent:
        \begin{enumerate}
            \item \(p\) ist eine Primzahl.
            \item \(\forall a,b\in\mathbb{N} (p|ab \Rightarrow p|a \vee p|b)\).
        \end{enumerate}
    \item \textbf{Eindeutigkeit der Primfaktorzerlegung (Fundamentalsatz der Arithmetik).} Jede \(n>1\) besitzt eine Darstellung
        \[
            n=\prod_{i=1}^k p_i^{\alpha_i},
        \]
        mit Primzahlen \(p_1<\dots<p_k\) und Exponenten \(\alpha_i\in\mathbb{N}\). Diese Darstellung ist bis auf die Reihenfolge eindeutig.

    \item \textbf{Primfaktorzerlegung und ggT.} Für zwei Zahlen
        \[
            a=\prod_{i=1}^n p_i^{\alpha_i},\qquad b=\prod_{i=1}^m p_i^{\beta_i},
        \]
        gilt insbesondere
        \[
            \operatorname{ggT}(a,b)=\prod_{i=1}^{\min(n,m)} p_i^{\min(\alpha_i,\beta_i)}.
        \]
        Beispiel: 
        \(\operatorname{ggT}(20,25)=\operatorname{ggT}(2^2\cdot 3^0\cdot 5^1, 2^0\cdot 3^0\cdot 5^2) \Rightarrow 2^0\cdot 3^0\cdot 5^1=5\).
\end{itemize}
\section{Modulare Arithmetik}

\subsection{Kongruenzrelation und Restklassen}
Für $n \in \mathbb{N}$ definiert man auf $\mathbb{Z}$ die Kongruenzrelation
\[
r \equiv_n s \;\Longleftrightarrow\; n \mid (r-s).
\]
Die Äquivalenzklasse von $z \in \mathbb{Z}$ heisst \emph{Restklasse} und wird mit
\[
[z]_n = \{z + kn \mid k \in \mathbb{Z}\}
\]
bezeichnet.
\begin{itemize}
    \item Abkürzend bezeichnet man \([z]_n\) mit \([z]\) oder \(\bar{z}\), wenn \(n\) aus dem Kontext klar ist.
    \item Jede ganze Zahl ist modulo $n$ eindeutig zu einer Zahl aus $\{0,\dots,n-1\}$ kongruent.
    \item Der Kleinste nicht negative Vertreter einer Restklasse $[z]_n$ wird als \emph{Kanonischer Vertreter} bezeichnet und ist gegeben durch $\operatorname{mod}(z,n)$.
\end{itemize}

\subsection{Rechnen mit Restklassen}
Die Menge der Restklassen modulo $n$ ist
\[
\mathbb{Z}/n = \{[0]_n,[1]_n,\dots,[n-1]_n\}.
\]
Addition und Multiplikation sind wohldefiniert durch
\[
[x]_n + [y]_n := [x+y]_n, 
\qquad
[x]_n \cdot [y]_n := [xy]_n.
\]

\subsection{Additive Inverse und lineare Gleichungen}
Jedes Element $[x]_n \in \mathbb{Z}/n$ besitzt ein additives Inverses $[-x]_n$
mit
\[
[x]_n + [-x]_n = [0]_n.
\]
Daher sind Gleichungen der Form
\[
a + x = b
\]
in $\mathbb{Z}/n$ für alle $a,b$ stets lösbar.

\subsection{Multiplikative Inverse}
Ein Element $[x]_n \in \mathbb{Z}/n$ besitzt genau dann ein multiplikatives
Inverses, wenn
\[
\operatorname{ggT}(n,x) = 1
\]
gilt. Insbesondere besitzt jedes Element ausser $[0]_n$ ein multiplikatives Inverses genau dann, wenn $n$ eine Primzahl ist.

\subsubsection{Multiplikatives Inverses berechnen}
Das multiplikative Inverse von $[a]_n$ kann mit dem erweiterten Euklidischen Algorithmus
berechnet werden,
indem man Zahlen $x,y \in \mathbb{Z}$ mit
\[ax + ny = 1
\]findet. Dann ist $[x]_n$ das gesuchte Inverse.

\subsection{Chinesischer Restsatz}
Sind $n_1,\dots,n_k \in \mathbb{N}_{>1}$ paarweise teilerfremd, so besitzt das Gleichungssystem simultaner
Kongruenzen
\begin{align*}
    x &\equiv_{n_1} y_1 \\
    x &\equiv_{n_2} y_2 \\
    &\vdots \\
    x &\equiv_{n_k} y_k
\end{align*}
eine eindeutige Lösung in $\mathbb{Z}/(n_1 \cdots n_k)$.

\subsubsection{Lösen simultaner Kongruenzen}
Die Lösung kann konstruiert werden, indem man die einzelnen Kongruenzen löst und die Lösungen dann kombiniert.
Man definiert
\[N = n_1 \cdots n_k,
\quad N_i = \frac{N}{n_i},
\]
und bestimmt die multiplikativen Inversen $M_i$ von $N_i$ modulo $n_i$.
Die Lösung des Gleichungssystems ist dann gegeben durch
\[x \equiv_N \sum_{i=1}^k y_i N_i M_i.\]

\subsection{Kleiner Satz von Fermat}
Ist $p$ eine Primzahl und $p \nmid a$, so gilt
\[
a^{p-1} \equiv_p 1.
\]

\subsection{Eulersche Phi-Funktion}
Für $n \in \mathbb{N}$ ist die Eulersche Phi-Funktion
\[\varphi(n) = |\{k \in \mathbb{N} \mid 1 \le k \le n, \operatorname{ggT}(k,n) = 1\}|
\]die Anzahl der positiven ganzen Zahlen bis $n$, die zu $n$ teilerfremd sind.
\begin{itemize}
    \item Ist $p$ eine Primzahl, so gilt $\varphi(p) = p-1$.
    \item Für $k \ge 1$ gilt $\varphi(p^k) = p^k - p^{k-1} = p^{k-1}(p-1)$.
    \item Ist $m,n \in \mathbb{N}$ mit $\operatorname{ggT}(m,n) = 1$, so gilt
\[
\varphi(mn) = \varphi(m) \cdot \varphi(n).
\]
    \item Für die Primfaktorzerlegung $n = p_1^{k_1} \cdots p_r^{k_r}$ gilt
\[\varphi(n) = n \left(1 - \frac{1}{p_1}\right) \cdots \left(1 - \frac{1}{p_r}\right).
\]
\end{itemize}


\vfill

\rule{0.3\linewidth}{0.25pt}
\scriptsize

By Justin Iven Müller (2025)
\href{https://github.com/JustinIven/zhaw-cheatsheets}{github.com/JustinIven/zhaw-cheatsheets}


\end{multicols*}
\end{document}
