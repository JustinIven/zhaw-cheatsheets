\documentclass[10pt,landscape]{article}
\usepackage{multicol}
\usepackage{calc}
\usepackage{ifthen}
\usepackage[landscape, a4paper, top=3mm, left=3mm, right=3mm, bottom=3mm]{geometry}
\usepackage{hyperref}
\usepackage{amsmath, amssymb}
\usepackage{enumitem}
\usepackage{tikz}
\usetikzlibrary{shapes,positioning,arrows,fit,calc,graphs,graphs.standard}
\usepackage[ngerman]{babel}



\begin{document}
% Turn off header and footer
\pagestyle{empty}
 

% Redefine section commands to use less space
\makeatletter
\renewcommand{\section}{\@startsection{section}{1}{0mm}%
                                {-1ex plus -.5ex minus -.2ex}%
                                {0.5ex plus .2ex}%x
                                {\normalfont\large\bfseries}}
\renewcommand{\subsection}{\@startsection{subsection}{2}{0mm}%
                                {-1explus -.5ex minus -.2ex}%
                                {0.5ex plus .2ex}%
                                {\normalfont\normalsize\bfseries}}
\renewcommand{\subsubsection}{\@startsection{subsubsection}{3}{0mm}%
                                {-1ex plus -.5ex minus -.2ex}%
                                {1ex plus .2ex}%
                                {\normalfont\small\bfseries}}
\makeatother

% Remove indentation in itemize/enumerate and set left margin to zero
\setlist[itemize]{leftmargin=*}

% Don't print section numbers
\setcounter{secnumdepth}{0}

% Set font size and alignment for the whole document
\raggedright
\footnotesize

% Set column separation
% \setlength{\columnseprule}{1pt}
% \def\columnseprulecolor{\color{blue}}

% TikZ settings for Venn diagrams
\def\firstcircle{(0,0) circle (1.5cm)}
\def\secondcircle{(0:2cm) circle (1.5cm)}

\tikzset{filled/.style={fill=circle area, draw=circle edge, thick},
    outline/.style={draw=circle edge, thick}}

\definecolor{myblue}{cmyk}{1,.72,0,.38}

\colorlet{circle edge}{myblue}
\colorlet{circle area}{myblue!5}

\begin{multicols*}{4}

\begin{center}
     \Large{\textbf{Diskrete Mathematik}} \\
\end{center}

\section{Zahlenmengen}
\begin{tabular}{@{}ll@{}}
\(\mathbb{N}\) & natürliche Zahlen \\
\(\mathbb{N}_0\) & natürliche Zahlen mit 0 \\
\(\mathbb{Z}\) & ganze Zahlen \\
\(\mathbb{Q}\) & rationale Zahlen \\
\(\mathbb{R}\) & reelle Zahlen \\
\(\mathbb{C}\) & komplexe Zahlen \\
\end{tabular}

\section{Aussagenlogik}
\begin{tabular}{@{} p{1.2cm}|p{4.5cm}| @{}}
Aussage & Ein Satz, der entweder wahr (w) oder falsch (f) ist. \\
Prädikat & Eine Aussage mit Variablen. {\it n}-stellige Prädikate. \\
\end{tabular}

\subsection{Grundidee}
Aus gegebenen Prädikaten/Aussagen lassen sich durch Junktoren neue Aussagen bilden. (z.\,B. Kombinationen mit \(\land,\lor,\lnot,\Rightarrow,\Leftrightarrow\)).

\subsection{Definitionen}
\begin{itemize}
  \item \textbf{Negation:} \(\lnot A\) ist genau dann wahr, wenn \(A\) falsch ist. (Doppelte Negation: \(A\Leftrightarrow\lnot\lnot A\).)
  \item \textbf{Konjunktion:} \(A\land B\) ist wahr genau dann, wenn \(A\) und \(B\) wahr sind. (assoziativ, kommutativ, idempotent)
  \item \textbf{Disjunktion:} \(A\lor B\) ist wahr, wenn mindestens eine der Aussagen wahr ist. (assoziativ, kommutativ, idempotent)
  \item \textbf{Implikation:} \(A\Rightarrow B\) ist äquivalent zu \(\lnot A\lor B\). (Kontraposition: \(A\Rightarrow B \Leftrightarrow \lnot B\Rightarrow\lnot A\).)
  \item \textbf{Äquivalenz:} \(A\Leftrightarrow B\) genau dann, wenn \(A\Rightarrow B \land B\Rightarrow A\).
\end{itemize}

\subsection{Wichtige Regeln}
\begin{itemize}
  \item \textbf{De Morgan:}\\ 
    \(\lnot(A\land B)\Leftrightarrow \lnot A\lor\lnot B\) 
    \(\lnot(A\lor B)\Leftrightarrow\lnot A\land\lnot B\)
  \item \textbf{Distributivität:}
    \(A\land(B\lor C)\Leftrightarrow (A\land B)\lor(A\land C)\)
    \(A\lor(B\land C)\Leftrightarrow (A\lor B)\land(A\lor C)\)
  \item \textbf{Syntaktische Bindung:} \(\lnot\) bindet stärker als \(\land,\lor\); diese binden stärker als \(\Rightarrow,\Leftrightarrow\).
  \item \textbf{Modus Ponens:} Aus \(A \land (A\Rightarrow B\)) folgt \(B\).
  \item \textbf{Transitivität:} Aus \((A\Rightarrow B) \land (B\Rightarrow C)\) folgt \(A\Rightarrow C\).
\end{itemize}

\subsection{Hinweis zur Redundanz}
Jeder Ausdruck mit den Junktoren $\lnot,\land,\lor,\Rightarrow$ lässt sich ausschliesslich mit \(\lnot\) und \(\lor\) darstellen. z.B.
$$A\land B \Leftrightarrow \lnot(\lnot A\lor\lnot B)$$

\section{Quantoren}
Quantoren dienen zur Formalisierung von Aussagen wie:
\begin{itemize}
  \item $\forall x\,A(x)$: \emph{Für alle $x$ gilt $A(x)$}
  \item $\exists x\,A(x)$: \emph{Es existiert ein $x$ mit $A(x)$}
\end{itemize}

Mehrere gleichartige Quantoren:
$$\forall x,y\;A(x,y) \quad\text{statt}\quad \forall x\,\forall y\;A(x,y)$$


\subsection{Eingeschränkte Quantoren}
$$\forall x \in M\,A(x): \text{Für alle }x\in M \text{ gilt }A(x)$$
$$\exists x \in M\,A(x): \text{Es gibt }x\in M \text{ mit }A(x)$$

Auch möglich mit Relationen:
$$\forall x < y\,A(x) \quad \text{oder} \quad \exists x \le y\,A(x)$$

\subsection{Als Junktoren}
Für endliche Mengen $M = \{x_1, \dots, x_n\}$ gilt:
$$\forall x \in M\,A(x) \Leftrightarrow A(x_1)\land \dots \land A(x_n)$$
$$\exists x \in M\,A(x) \Leftrightarrow A(x_1)\lor \dots \lor A(x_n)$$

\subsection{Als Makros}
$$\exists x \in M\,A(x) \Leftrightarrow \exists x\,(x \in M \land A(x))$$
$$\forall x \in M\,A(x) \Leftrightarrow \forall x\,(x \in M \Rightarrow A(x))$$

\subsection{Zusammenhang mit Junktoren}
$$\neg \forall x\,A(x) \Leftrightarrow \exists x\,\neg A(x)
\quad\text{und}\quad
\neg \exists x\,A(x) \Leftrightarrow \forall x\,\neg A(x)$$
$$\forall x\,(A(x)\land B(x)) \Leftrightarrow (\forall x\,A(x)) \land (\forall x\,B(x))$$
$$\exists x\,(A(x)\lor B(x)) \Leftrightarrow (\exists x\,A(x)) \lor (\exists x\,B(x))$$

\subsection{Leere Quantoren}
Wenn $x$ in $B$ nicht vorkommt:
$$\forall x\,B \Leftrightarrow B, \quad \exists x\,B \Leftrightarrow B$$

\section{Mengen}
\begin{itemize}
  \item \textbf{Menge / Element:} Eine Menge fasst mathematische Objekte (Elemente) zu einem Ganzen zusammen. Für Menge \(X\) und Element \(y\) gilt \(y\in X\) bzw.\ \(y\notin X\).
  \item \textbf{Aufzählende Schreibweise:} \(\{x_1,\dots,x_n\}\) bezeichnet die Menge, die genau die genannten Elemente enthält. Die leere Menge heisst \(\varnothing\).
  \item \textbf{Extensionalitätsprinzip:} Zwei Mengen sind genau dann gleich, wenn sie dieselben Elemente haben:
  \[ A=B \iff \forall x\,(x\in A \Leftrightarrow x\in B). \]
  \item \textbf{Teilmenge:} \(A\subseteq B\) genau dann, wenn \(\forall x\,(x\in A \Rightarrow x\in B)\). Ist \(A\subseteq B\) und \(A\neq B\), so ist \(A\) eine \emph{echte} Teilmenge, geschrieben \(A\subset B\).
  \item \textbf{Folgerungen:} Mengen sind ungeordnet; Mehrfachaufzählung desselben Elements ändert die Menge nicht. Für jede Menge \(A\) gilt \(\varnothing\subseteq A\).
\end{itemize}

\subsection{Eindeutigkeit der leeren Menge}
Seien \(e_1,e_2\) leere Mengen. Dann ist für alle \(x\) die Aussage \(x\in e_1\) falsch, also ist die Implikation \(x\in e_1\Rightarrow x\in e_2\) wahr; somit \(e_1\subseteq e_2\). Analog \(e_2\subseteq e_1\). Nach Extensionalität folgt \(e_1=e_2\).


\subsection{Aussonderungsprinzip}
Ist \(A\) eine Menge und \(E(x)\) eine Eigenschaft, dann gilt:
\[
\{x \in A \mid E(x)\} = \text{Menge aller } x \in A \text{ mit Eigenschaft } E(x).
\]
\[
a \in \{x \in A \mid E(x)\} \iff a \in A \land E(a)
\]

\textbf{Beispiele:}
\begin{itemize}
  \item Gerade Zahlen: \(\{x \in \mathbb{N} \mid \exists y \in \mathbb{N} (x = 2y)\}\)
  \item Zahlen \(> 17\): \(\{x \in \mathbb{N} \mid x > 17\}\)
  \item Alle ausser \(22\): \(\{x \in \mathbb{N} \mid x \neq 22\}\)
\end{itemize}

\subsection{Ersetzungsprinzip}
Ist \(A\) eine Menge und \(t(x)\) ein Ausdruck, so gilt:
\[ \{t(x) \mid x \in A\} = \text{Menge aller Werte von } t(x) \text{ mit } x \in A. \]
\[ a \in \{t(x) \mid x \in A\} \iff \exists x \in A (a = t(x)) \]

\textbf{Beispiele:}
\begin{itemize}
  \item Quadratzahlen: \(\{x^2 \mid x \in \mathbb{N}\}\)
  \item Ungerade Zahlen: \(\{2x + 1 \mid x \in \mathbb{N}\}\)
  \item Rationale Zahlen: \(\left\{\frac{a}{b} \mid a,b \in \mathbb{Z},\, b \neq 0\right\}\)
  \item Anfangsabschnitte von \(\mathbb{N}\): \(\{\{x \in \mathbb{N} \mid x < y\} \mid y \in \mathbb{N}\}\)
\end{itemize}




\subsection{Vereinigung}
Die Vereinigung von zwei Mengen beinhaltet genau die Elemente, die in mindestens einer der beiden Mengen enthalten sind:
\[
A\cup B:=\{x\mid x\in A\vee x\in B\}.
\]

\subsection{Schnitt}
Die Schnittmenge von zwei Mengen beinhaltet genau die Elemente, die in beiden Mengen enthalten sind:
\[
A\cap B:=\{x\mid x\in A\land x\in B\}.
\]

\subsection{Allgemeine Vereinigung / Schnitt}
Sei \(I\) eine beliebige Indexmenge (z.\,B. \(I = \{1,2,\dots,n\}\) oder \(I = \mathbb{N}\)).  
Für jedes \(i \in I\) sei \(A_i\) eine Menge.

\subsubsection{Allgemeine Vereinigung}
\(x\) gehört zur Vereinigung genau dann, wenn es in \emph{mindestens einer} der Mengen \(A_i\) enthalten ist.
\[
\bigcup_{i \in I} A_i := \{\, x \mid \exists i \in I : x \in A_i \,\}.
\]


\subsubsection{Allgemeiner Schnitt}
\(x\) gehört zum Schnitt genau dann, wenn es in \emph{allen} Mengen \(A_i\) enthalten ist.
\[
\bigcap_{i \in I} A_i := \{\, x \mid \forall i \in I : x \in A_i \,\}.
\]

\subsection{Differenz}
Die Differenz von zwei Mengen beinhaltet genau die Elemente, die in der ersten Menge, aber nicht in der zweiten Menge enthalten sind:
\[
A\setminus B:=\{x\in A\mid x\notin B\}.
\]

\subsection*{Disjunkte Mengen}
Zwei Mengen \(A\) und \(B\) heissen disjunkt, wenn sie keine gemeinsamen Elemente besitzen.
\[
A \cap B = \varnothing.
\]

\subsection{Paarweise disjunkt}
Eine Familie von Mengen \((A_i)_{i \in I}\) heisst paarweise disjunkt, wenn keine zwei verschiedenen Mengen ein gemeinsames Element haben. Es gilt:
\[
\forall i,j \in I \; (i \neq j \Rightarrow A_i \cap A_j = \varnothing).
\]

\subsection{Wichtige Eigenschaften}
Für beliebige Mengen \(A,B,C\) gelten:
\begin{itemize}
  \item Idempotenz: \(A\cup A=A,\; A\cap A=A\).
  \item Kommutativität: \(A\cup B=B\cup A,\; A\cap B=B\cap A\).
  \item Assoziativität: \(A\cup(B\cup C)=(A\cup B)\cup C\) und analog für \(\cap\).
  \item Teilmengen: \(A\subseteq A\cup B\) und \(A\cap B\subseteq A\).
  \item Distributivität:
  \[
  A\cup(B\cap C)=(A\cup B)\cap(A\cup C),
  \]
  \[
  A\cap(B\cup C)=(A\cap B)\cup(A\cap C).
  \]
  \item De Morgansche Regeln:
  \[
  C\setminus(A\cap B)=(C\setminus A)\cup(C\setminus B),
  \]
  \[
  C\setminus(A\cup B)=(C\setminus A)\cap(C\setminus B).
  \]
\end{itemize}

\subsection{Venn-Diagramm}
% Set A and B
\begin{tikzpicture}[scale=.45]
    \begin{scope}
        \clip \firstcircle;
        \fill[filled] \secondcircle;
    \end{scope}
    \draw[outline] \firstcircle node {$A$};
    \draw[outline] \secondcircle node {$B$};
    \node[anchor=south] at (current bounding box.north) {$A \cap B$};
\end{tikzpicture}
%Set A or B but not (A and B) also known a A xor B
\begin{tikzpicture}[scale=.45]
    \draw[filled, even odd rule] \firstcircle node {$A$}
                                 \secondcircle node{$B$};
    \node[anchor=south] at (current bounding box.north) {$\overline{A \cap B}$};
\end{tikzpicture}
% Set A or B
\begin{tikzpicture}[scale=.45]
    \draw[filled] \firstcircle node {$A$}
                  \secondcircle node {$B$};
    \node[anchor=south] at (current bounding box.north) {$A \cup B$};
\end{tikzpicture}
% Set A but not B
\begin{tikzpicture}[scale=.45]
    \begin{scope}
        \clip \firstcircle;
        \draw[filled, even odd rule] \firstcircle node {$A$}
                                     \secondcircle;
    \end{scope}
    \draw[outline] \firstcircle
                   \secondcircle node {$B$};
    \node[anchor=south] at (current bounding box.north) {$A\setminus B$};
\end{tikzpicture}

\subsection{Potenzmenge}
Für eine Menge \(A\) bezeichnet die Potenzmenge \( \mathcal{P}(A) \) die Menge aller Teilmengen von \(A\):
\[
\mathcal{P}(A):=\{X \mid X\subseteq A\}
\]

\textbf{Beispiele:}
\begin{align*}
\mathcal{P}(\{1,2\}) &= \{\varnothing,\{1\},\{2\},\{1,2\}\},\\
\mathcal{P}(\varnothing) &= \{\varnothing\},\\
\mathcal{P}(\{\{a\}\}) &= \{\varnothing,\{\{a\}\}\}.
\end{align*}

\textbf{Eigenschaften:}
\begin{itemize}
  \item \(A\in\mathcal{P}(A)\) und \(\varnothing\in\mathcal{P}(A)\).
  \item Aus \(A\subseteq B\) folgt \(\mathcal{P}(A)\subseteq\mathcal{P}(B)\).
  \item Für die leere Menge gilt \(\mathcal{P}(\varnothing)=\{\varnothing\}\neq\varnothing\).
  \item \(\mathcal{P}(A\cap B)=\mathcal{P}(A)\cap\mathcal{P}(B)\)
  \item \(\mathcal{P}(A\cup B)\supseteq \mathcal{P}(A)\cup\mathcal{P}(B)\)
\end{itemize}




\vspace{1cm}

\rule{0.3\linewidth}{0.25pt}
\scriptsize

Copyright \copyright\ 2025 Justin Iven Müller

\href{https://github.com/JustinIven/zhaw-cheatsheets}{github.com/JustinIven/zhaw-cheatsheets}


\end{multicols*}
\end{document}
