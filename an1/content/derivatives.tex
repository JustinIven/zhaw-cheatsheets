\section{Differentialrechnung}
Sei \(f(x) : D \to \mathbb{R}\) eine Funktion. Die Ableitung \(f'(x_0)\) beschreibt die Änderungsrate der Funktion an der Stelle \(x_0\). Sie wird definiert als der Grenzwert des Differenzenquotienten:
\[
    f'(x_0) = \lim_{\Delta x \to 0} \frac{f(x_0 + \Delta x) - f(x_0)}{\Delta x}
\]

\subsection{Ableitungsregeln}
\begin{itemize}
    \item \textbf{Faktorregel:} \((c \cdot f)'(x) = c \cdot f'(x)\)
    \item \textbf{Summenregel:} \((f + g)'(x) = f'(x) + g'(x)\)
    \item \textbf{Produktregel:} \((f \cdot g)'(x) = f'(x) \cdot g(x) + f(x) \cdot g'(x)\)
    \item \textbf{Quotientenregel:} \(\left(\frac{f}{g}\right)'(x) = \frac{f'(x) \cdot g(x) - f(x) \cdot g'(x)}{(g(x))^2}\)
    \item \textbf{Kettenregel:} \((f \circ g)'(x) = f'(g(x)) \cdot g'(x)\)
\end{itemize}

\subsection{Wichtige Ableitungen}
\begin{align*}
    (x^n)' &= n \cdot x^{n-1} \quad\text{ für } n \in \mathbb{R} \\
    (e^x)' &= e^x \\
    (\ln x)' &= \frac{1}{x} \quad \text{ für } x > 0 \\
    (\log_a x)' &= \frac{1}{x \ln a} \quad \text{ für } x > 0, a > 0, a \neq 1 \\
    (\sin x)' &= \cos x \\
    (\cos x)' &= -\sin x \\
\end{align*}

\subsection{Differenzierbarkeit}
Eine Funktion \(f(x)\) ist an der Stelle \(x_0\) differenzierbar, wenn die links- und rechtsseitigen Ableitungen übereinstimmen. Differenzierbarkeit impliziert Stetigkeit, aber nicht umgekehrt.\\
\emph{Definition:} Eine Funktion heisst differenzierbar, wenn die Ableitung an jeder Stelle ihres Definitionsbereichs existiert.

\subsection{Tangenten- und Normalengleichung}
Die Tangente an die Funktion \(f(x)\) an der Stelle \(x_0\) hat die Gleichung:
\[
    y = f'(x_0)(x - x_0) + f(x_0)
\]
Die Normalenlinie, die senkrecht zur Tangente steht, hat die Gleichung:
\[
    y = -\frac{1}{f'(x_0)}(x - x_0) + f(x_0)
\]

\subsection{Newton-Verfahren}
Das Newton-Verfahren ist ein iteratives Verfahren zur Bestimmung von Nullstellen einer Funktion. Gegeben sei eine Funktion \(f(x)\) und eine Näherung \(x_0\) für eine Nullstelle (Fixpunkt). Die Iterationsformel lautet:
\[
    x_{n+1} = x_n - \frac{f(x_n)}{f'(x_n)}
\]
Die Konvergenz des Verfahrens hängt von der Wahl der Startnäherung und den Eigenschaften der Funktion ab.

