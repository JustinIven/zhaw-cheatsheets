\section{Ableitungen}
Eine Ableitung beschreibt die Änderung (Steigung in der Ebene) einer Funktion. Die Ableitung einer Funktion $f(x)$ wird als $f'(x)$ oder $\frac{df}{dx}$ notiert.

\subsection{Regeln der Differentiation}
Es gibt einige grundlegende Regeln zur Ableitung von Funktionen:

\begin{itemize}
    \item \textbf{Konstantenregel:} \(f(x) = c \Rightarrow f'(x) = 0\), wobei \(c\) eine Konstante ist.
    \item \textbf{Potenzregel:} \(f(x) = x^n \Rightarrow f'(x) = n \cdot x^{n-1}\).
    \item \textbf{Summenregel:} \(f(x) = g(x) + h(x) \Rightarrow f'(x) = g'(x) + h'(x)\).
    \item \textbf{Produktregel:} \(f(x) = g(x) \cdot h(x) \Rightarrow f'(x) = g'(x) \cdot h(x) + g(x) \cdot h'(x)\).
    \item \textbf{Quotientenregel:} \(f(x) = \frac{g(x)}{h(x)} \Rightarrow f'(x) = \frac{g'(x) \cdot h(x) - g(x) \cdot h'(x)}{h(x)^2}\).
\end{itemize}

\subsection{Höhere Ableitungen}
Die zweite Ableitung $f''(x)$ ist die Ableitung der ersten Ableitung und beschreibt die Krümmung der Funktion. Höhere Ableitungen können ebenfalls gebildet werden und liefern Informationen über das Verhalten der Funktion.

\subsection{Anwendungen der Ableitung}
Ableitungen finden in vielen Bereichen Anwendung, z.B. in der Physik zur Beschreibung von Geschwindigkeiten und Beschleunigungen oder in der Wirtschaft zur Analyse von Kosten- und Erlösfunktionen.