\section{Polynome}
Ein Polynom ist eine Funktion der Form
\[
    f(x) = a_n x^n + a_{n-1} x^{n-1} + \ldots + a_1 x + a_0
\]
mit $a_n, a_{n-1}, \ldots, a_1, a_0 \in \mathbb{R}$ und $a_n \neq 0$.
Der höchste Exponent $n$ heisst \emph{Grad} des Polynoms.

\subsection{Horner-Schema}
Das Horner-Schema ist eine effiziente Methode zur Auswertung von Polynomen. Es hat die Form:
\[
    f(x) = (((a_n x + a_{n-1}) x + a_{n-2}) x + \ldots + a_1) x + a_0
\]
...
\subsection{Nullstellen von Polynomen}
Ein Polynom von Grad $n$ hat höchstens $n$ reelle Nullstellen.
...

\subsection{Polynomdivision}
...