\section{Polynomfunktionen}
Ein Polynom ist eine Funktion der Form
\[
    f(x) = a_n x^n + a_{n-1} x^{n-1} + \ldots + a_1 x + a_0
\]
mit $a_n, a_{n-1}, \ldots, a_1, a_0 \in \mathbb{R}$ und $a_n \neq 0$.
Der höchste Exponent $n$ heisst \emph{Grad} des Polynoms.

\subsection{Horner-Schema}
Das Horner-Schema ist eine effiziente Methode zur Auswertung von Polynomen. Es hat die Form:
\[
    f(x) = (((a_n x + a_{n-1}) x + a_{n-2}) x + \ldots + a_1) x + a_0
\]

Linearfaktorisierung eines Polynoms nach dem Horner-Schema:
\[
    f(x) = 2x^3-12x^2+22x-12
\]

\begin{center}
  \polyhornerscheme[x=2, showvar=true, stage=8, tutor=true, tutorlimit=8, tutorstyle=\color{lightgray}]{2x^3-12x^2+22x-12}
\end{center}
\[
    f(x) = (2x^2 - 8x + 6)(x - 2)
\]

\subsection{Nullstellen von Polynomen}
Ein Polynom von Grad \(n\) hat höchstens \(n\) reelle Nullstellen.

\subsection{\(m\)-fache Nullstelle}
\(x_0\) ist eine \(m\)-fache Nullstelle des Polynoms \(f(x)\), wenn \(f(x)\) in der Form
\[
    f(x) = (x - x_0)^m \cdot g(x)
\]
geschrieben werden kann, wobei \(g(x)\) ein Polynom ist, das \(x_0\) nicht als Nullstelle hat.

\subsection{Polynomdivision}
Mit dem Horner-Schema können nur Linearfaktoren \((x - x_0)\) abgeteilt werden. Für die Division durch Polynome höheren Grades wird die Polynomdivision verwendet:
\[
    f(x) = g(x) \cdot Q(x) + R(x)
\]
wobei \(Q(x)\) der Quotient und \(R(x)\) der Rest ist. Der Grad des Rests ist kleiner als der Grad des Divisors \(g(x)\).
\begin{center}
    \tiny
    \polylongdiv[style=C, div=:]{x^3-6x^2+11x-6}{x^2+x-4}
\end{center}