
\section{Funktionen}
Eine Funktion $f$ ordnet jedem Element $x$ aus einem Definitionsbereich $D$ genau einen Wert $f(x)$ zu.

\subsection*{Grundbegriffe}
\begin{itemize}
\item \textbf{Definitionsbereich $D$}: alle $x$, für die $f(x)$ definiert ist.
\item \textbf{Wertebereich $W$}: alle Werte $f(x)$, die tatsächlich auftreten.
\item \textbf{Graph}: Menge aller Punkte $(x,f(x))$ im Koordinatensystem.
\end{itemize}

\subsection*{Wichtige Eigenschaften}
\textbf{Gerade Funktion}: $f(-x)=f(x)$. Graph ist symmetrisch zur $y$-Achse.  
\textbf{Ungerade Funktion}: $f(-x)=-f(x)$. Graph ist punktsymmetrisch zum Ursprung.

\subsection*{Umkehrfunktionen}
$f$ besitzt eine Umkehrfunktion $f^{-1}$ genau dann, wenn $f$ \textbf{injektiv} ist.  
\[
(f^{-1})'(y)=\frac{1}{f'(x)}\quad\text{mit }y=f(x).
\]
\textit{Erklärung: Eine Umkehrfunktion ``dreht'' die Zuordnung um; ihre Steigung ist der Kehrwert der ursprünglichen.}

\section{Polynome und Algebraische Werkzeuge}
Ein Polynom vom Grad $n$ hat höchstens $n$ Nullstellen.

\subsection*{Horner-Schema}
Effiziente Berechnung von $f(x_0)$ und Polynomdivision durch $(x-x_0)$.

\subsection*{Zerlegungssatz}
Hat $f(x_0)=0$, so lässt sich schreiben:
\[
f(x) = (x-x_0)\cdot q(x).
\]

\subsection*{Wichtige Summenformeln}
\[
\sum_{k=1}^n k = \frac{n(n+1)}{2},\qquad
\sum_{k=1}^n k^2 = \frac{n(n+1)(2n+1)}{6}.
\]

\textit{Erklärung: Diese Formeln ermöglichen es, Summen geschlossener auszudrücken.}

\section{Differentialrechnung}
\subsection*{Definition der Ableitung}
\[
f'(x)=\lim_{h\to0}\frac{f(x+h)-f(x)}{h}.
\]

\textit{Interpretation: $f'(x)$ beschreibt die Steigung der Tangente an den Graphen an der Stelle $x$.}

\subsection*{Ableitungsregeln}
Linearität:
\[
(c\cdot f)' = c\cdot f', \quad (f+g)' = f'+g'
\]

Produktregel:
\[
(fg)' = f'g + fg'
\]

Quotientenregel:
\[
\left(\frac{u}{v}\right)' = \frac{u'v-uv'}{v^2}
\]

Kettenregel (Zusammensetzung $F(u(x))$):
\[
(F\circ u)' = F'(u(x)) \cdot u'(x)
\]

\subsection*{Tangente}
\[
T(x)=f(x_0)+f'(x_0)(x-x_0).
\]

\textit{Bedeutung: Lokale lineare Annäherung an den Graphen.}

\subsection*{Extremstellen (Hoch-/Tiefpunkte)}
\[
f'(x_0)=0.
\]
Zur Klassifikation:
\[
\begin{cases}
f''(x_0) > 0 & \text{Minimum}\\
f''(x_0) < 0 & \text{Maximum}\\
f''(x_0) = 0 & \text{keine eindeutige Aussage}
\end{cases}
\]

\section{Integralrechnung}
\subsection*{Definition als Grenzwert}
\[
\int_a^b f(x)\,dx = \lim_{n\to\infty}\sum_{k=1}^n f(x_k)\Delta x.
\]

\textit{Interpretation: Fläche unter dem Graphen zwischen $a$ und $b$.}

\subsection*{Hauptsatz der Differential- und Integralrechnung}
Ist $F$ eine Stammfunktion von $f$, dann:
\[
\int_a^b f(x)\,dx = F(b)-F(a).
\]

\subsection*{Wichtige Integrationsregeln}
Lineare Operationen:
\[
\int(c\cdot f)dx=c\cdot\int f dx,\qquad \int(f+g)dx=\int f dx + \int g dx.
\]

Potenzregel ($n\neq -1$):
\[
\int x^n dx = \frac{x^{n+1}}{n+1} + C.
\]

Substitution:
\[
\int f(u(x))u'(x)\,dx = \int f(u)\,du.
\]

\section{Folgen und Reihen}
\subsection*{Grenzwert einer Folge}
Eine Folge $(a_n)$ konvergiert gegen $L$, wenn:
\[
\lim_{n\to\infty} a_n = L.
\]

\textit{Bedeutung: Für grosse $n$ wird $a_n$ beliebig nahe an $L$.}

\subsection*{Reihen}
Eine Reihe ist die Summe einer Folge:
\[
\sum_{n=1}^{\infty} a_n.
\]

\textbf{Geometrische Reihe:}
\[
\sum_{k=0}^\infty q^k = \frac{1}{1-q} \quad \text{für } |q|<1.
\]

\section{Grenzwerte und Stetigkeit}
\[
\lim_{x\to x_0} f(x)=f(x_0) \quad\Rightarrow\quad f \text{ ist stetig in } x_0.
\]

\textit{Bedeutung: Keine Sprungstellen; Graph kann ohne Absetzen gezeichnet werden.}
