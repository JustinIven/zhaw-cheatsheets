\section{Integralrechnung}
\subsection{Unbestimmte Integrale}
Sei \(f:\mathbb{R}\to\mathbb{R}\) eine Funktion.
Das \textbf{unbestimmte Integral} von \(f\) ist definiert als
\[\int f(x)\,dx = F(x) + C,
\]
wobei \(F\) eine Stammfunktion von \(f\) ist, also \(F'(x) = f(x)\) für alle \(x \in \mathbb{R}\), und \(C\) eine Konstante ist.

\subsection{Bestimmte Integrale}
Sei \(f:[a,b]\to\mathbb{R}\) eine Funktion.
Das \textbf{bestimmte Integral} von \(f\) über das Intervall \([a,b]\) ist definiert als
\[\int_a^b f(x)\,dx = F(b) - F(a),
\]
wobei \(F\) eine beliebige Stammfunktion von \(f\) ist, also \(F'(x) = f(x)\) für alle \(x \in [a,b]\).
\begin{center}
\begin{tikzpicture}
\begin{axis}[
    axis lines=middle,
    xmin=-3, xmax=3,
    ymin=-10, ymax=10,
    xtick distance=1, ytick distance=4,
    scale=.45, yticklabel={\empty}, xticklabel={\empty}
    ]
    \path [name path=xaxis] (axis cs:-10,0) -- (axis cs:10,0);
    \addplot [domain=-2.5:2.5, samples=100, name path=f, thick, color=red!50]{-x^2 + 2*x + 5};
    \addplot[red!10, opacity=0.4] fill between[of=f and xaxis, soft clip={domain=-0.5:2}];
    \draw [dashed, opacity=0.4] (axis cs:{-0.5,0}) -- (axis cs:{-0.5,8});
    \draw [dashed, opacity=0.4] (axis cs:{2,0}) -- (axis cs:{2,8});
    \node at (axis cs: -0.5,-1) {\tiny$a$};
    \node at (axis cs: 2,-1) {\tiny$b$};
\end{axis}
\end{tikzpicture}
\end{center}

\subsection{Flächeninhalt unter einer Funktion}
Sei \(f:[a,b]\to\mathbb{R}\) eine Funktion mit \(f(x) \geq 0\) für alle \(x \in [a,b]\).
Der Flächeninhalt \(A\) unter der Funktion \(f\) über das Intervall \([a,b]\) ist definiert als
\[A = \int_a^b f(x)\,dx.
\]

Für den Fall, dass \(f(x) \leq 0\) für alle \(x \in [a,b]\), ist der Flächeninhalt \(A\) über der Funktion \(f\) definiert als
\[A = -\int_a^b f(x)\,dx.
\]

\emph{Im Allgemeinen:} Sei \(X\) die Menge aller Nullstellen von \(f\) im Intervall \([a,b]\), also \(X = \{x_1, x_2, \ldots, x_n\}\) mit \(a = x_0 < x_1 < \ldots < x_n = b\).
Dann ist der Flächeninhalt \(A\) zwischen der Funktion \(f\) und der \(x\)-Achse über das Intervall \([a,b]\) definiert als
\[A = \sum_{i=0}^{n-1} \left| \int_{x_i}^{x_{i+1}} f(x)\,dx \right|.\]
\begin{center}
\begin{tikzpicture}
\begin{axis}[
    axis lines=middle,
    xmin=-3, xmax=3,
    ymin=-10, ymax=10,
    xtick distance=1, ytick distance=4,
    scale=.45, yticklabel={\empty}, xticklabel={\empty}
    ]
    \addplot [domain=-2.5:2.5, samples=100, name path=f, thick, color=red!50]{3*x^3 - x^2 - 10*x};
    \path [name path=xaxis] (axis cs:-10,0) -- (axis cs:10,0);
    \draw [dashed, opacity=0.4] (axis cs:{-2,0}) -- (axis cs:{-2,-8});
    \addplot[red!10, opacity=0.4] fill between [
        of=f and xaxis, soft clip={domain=-2:2},
        split,
        every even segment/.style = {blue!20!white},
        every odd segment/.style ={red!20!white}];
    \node at (axis cs: -2,-1) {\tiny$x_0$};
    \node at (axis cs: 0,-1) {\tiny$x_1$};
    \node at (axis cs: 2,-1) {\tiny$x_2$};

\end{axis}
\end{tikzpicture}
\end{center}




\subsection{Integral zwischen zwei Funktionen}
Seien \(f,g:[a,b]\to\mathbb{R}\) zwei Funktionen.
Das \textbf{Integral zwischen zwei Funktionen} \(f\) und \(g\) über das Intervall \([a,b]\) ist definiert als
\[\int_a^b (f(x) - g(x))\,dx = F(b) - G(a),
\]
wobei \(F\) eine Stammfunktion von \(f\) und \(G\) eine Stammfunktion von \(g\) ist.

\begin{center}
\begin{tikzpicture}
\begin{axis}[
    axis lines=middle,
    xmin=-3, xmax=3,
    ymin=-10, ymax=10,
    xtick distance=1, ytick distance=4,
    scale=.45, yticklabel={\empty}, xticklabel={\empty}
    ]
    \addplot [domain=-2.5:2.5, samples=100, name path=f, thick, color=red!50]{3*x^3 - x^2 - 10*x};
    \addplot [domain=-2.5:2.5, samples=100, name path=g, thick, color=blue!50]{- x^2 + 2*x};
    \addplot[red!10, opacity=0.4] fill between[of=f and g, soft clip={domain=-2:2}];
    \draw [dashed, opacity=0.4] (axis cs:{-2,0}) -- (axis cs:{-2,-8});
    \node[color=red, font=\footnotesize] at (axis cs: -1.6,7) {\tiny$f(x)$};
    \node[color=blue, font=\footnotesize] at (axis cs: 1.1,2.2) {\tiny$g(x)$};
    \node at (axis cs: -2,-1) {\tiny$a$};
    \node at (axis cs: 2,-1) {\tiny$b$};
\end{axis}
\end{tikzpicture}
\end{center}

\subsection{Integrationsregeln}
\subsubsection{Konstantenregel}
Sei \(c \in \mathbb{R}\) eine Konstante. Dann gilt für das Integral einer konstanten Funktion
\[\int_a^b c\,dx = c \cdot (b-a).\]

\subsubsection{Summenregel}
Seien \(f,g:[a,b]\to\mathbb{R}\) zwei Funktionen. Dann gilt für das Integral der Summe von zwei Funktionen
\[\int_a^b (f(x) + g(x))\,dx = \int_a^b f(x)\,dx + \int_a^b g(x)\,dx.\]

\subsubsection{Polynomregel}
Sei \(n \in \mathbb{N}_0\). Dann gilt für das Integral einer Potenzfunktion
\[\int x^n\,dx = \frac{1}{n+1} x^{n+1} + C.\]

\subsection{Integrale von ausgewählten Funktionen}
Potentz- und Logarithmusfunktionen:
\begin{align*}
\int a^x\,dx &= \frac{a^x}{\ln(a)} + C \quad (a > 0, a \neq 1)\\
\int \ln(x)\,dx &= x \ln(x) - x + C \quad (x > 0)\\
\int \log_a(x)\,dx &= \frac{x \ln(x) - x}{\ln(a)} + C \quad (a > 0, a \neq 1, x > 0)
\end{align*}
Trigonometrische Funktionen:
\begin{align*}
\int \sin(x)\,dx &= -\cos(x) + C\\
\int \cos(x)\,dx &= \sin(x) + C\\
\int \tan(x)\,dx &= -\ln|\cos(x)| + C
\end{align*}